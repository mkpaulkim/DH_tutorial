\documentclass[t, aspectratio=169]{beamer}
\usepackage[utf8]{inputenc}

\usetheme[progressbar=frametitle]{Madrid}
%\usetheme{Antibes}
\usecolortheme{beaver}
\setbeamertemplate{frame numbering}[fraction]

\title{Multi-Wavelength Techniques in Digital Holography: Tutorial}
\author{Myung K Kim}
\institute{Dept of Physics, University of South Florida, Tampa, FL USA 33620}
\date{August 2022}


\begin{document}

\begin{frame}
	Optica: Digital Holography and Three-Dimensional Imaging\\
	Cambridge University, Cambridge, UK
	\titlepage
\end{frame}


\begin{frame}[allowframebreaks]{Table of Contents}
	\tableofcontents[hideallsubsections]
\end{frame}


\begin{frame}{abstract}
Basic principles and techniques will be described for multi-wavelength or multi-illumination-angle methods for extending capabilities of digital holography for surface profilomentry by optical phase unwrapping and topography/tomography by snthesized optical coherence.
\end{frame}


\section{holography background}
\begin{frame}{\secname}
	\tableofcontents[currentsection, hideothersubsections, sectionstyle=hide/hide]
\end{frame}


\subsection{holography cartoon}



\subsection{basic theoretical description of holography}



\subsection{theory of digital holography}



\subsection{highlights of DH techniques \& applications}



\section{multi-wavelength background}
\begin{frame}{\secname}
	\tableofcontents[currentsection, hideothersubsections, sectionstyle=hide/hide]
\end{frame}


\subsection{cartoon: two-wavelength interference}



\subsection{multi-wavelength analog holographic interference: examples}



\subsection{multi-wavelength interferometic methods: examples}



%\begin{frame}{multi-wavelength background}
%	\begin{itemize}
%		\item cartoon: two-wavelength interference
%		\item multi-wavelength analog holographic interference: examples
%		\item multi-wavelength interferometic methods: examples
%	\end{itemize}
%\end{frame}


\section{digital holography system}
\begin{frame}{\secname}
	\tableofcontents[currentsection, hideothersubsections, sectionstyle=hide/hide]
\end{frame}


\subsection{theory of digital holography}



\subsection{interferometer configurations}



\subsection{apparatus and hardware}



\subsection{software for experimental control}



\subsection{software for numerical calculation}



%\begin{frame}{digital holography system}
%	\begin{itemize}
%		\item theory of digital holography
%		\item interferometer configurations
%		\item apparatus and hardware
%		\item software for experimental control
%		\item software for numerical calculation
%	\end{itemize}
%\end{frame}


\section{2WQPM principles}
\begin{frame}{\secname}
	\tableofcontents[currentsection, hideothersubsections, sectionstyle=hide/hide]
\end{frame}

\subsection{2WQPM theory}



\subsection{2Wqpm simulation}


\subsection{2WQPM demonstration extp}

\begin{frame}{apparatus}

\end{frame}

%\begin{frame}{2WQPM theory}
%	\begin{itemize}
%		\item 2WQPM theory
%		\item simulation
%	\end{itemize}
%\end{frame}


%\begin{frame}{2WQPM demonstration expt}
%	\begin{itemize}
%		\item apparatus
%	\end{itemize}
%\end{frame}


\section{2WQPM techniques \& applications}
\begin{frame}{\secname}
	\tableofcontents[currentsection, hideothersubsections, sectionstyle=hide/hide]
\end{frame}




\section{MWQPM theory}
\begin{frame}{\secname}
	\tableofcontents[currentsection, hideothersubsections, sectionstyle=hide/hide]
\end{frame}




\section{MWQPM expt}
\begin{frame}{\secname}
	\tableofcontents[currentsection, hideothersubsections, sectionstyle=hide/hide]
\end{frame}




\section{WSDIH theory}
\begin{frame}{\secname}
	\tableofcontents[currentsection, hideothersubsections, sectionstyle=hide/hide]
\end{frame}




\section{WSDIH expt}
\begin{frame}{\secname}
	\tableofcontents[currentsection, hideothersubsections, sectionstyle=hide/hide]
\end{frame}




\section{multi-angle theory}
\begin{frame}{\secname}
	\tableofcontents[currentsection, hideothersubsections, sectionstyle=hide/hide]
\end{frame}




\section{multi-angle expt}
\begin{frame}{\secname}
	\tableofcontents[currentsection, hideothersubsections, sectionstyle=hide/hide]
\end{frame}




\section{references}



\frame{ }
\end{document}



%\section{history of optics: let there be light}
%
%\begin{frame}{history of optics}
%	\begin{itemize}
%		\item ancient theory of light and vision
%		\item development of early modern optics
%		\item geometrical optics
%		\item wave optics 
%		\item quantum optics
%	\end{itemize}
%	\vspace{10 mm}
%	$\Rightarrow$ \href{https://en.wikipedia.org/wiki/History_of_optics}{Wikipedia: History of optics}
%\end{frame}
%
%
%\begin{frame}[allowframebreaks]{ancient theory of light and vision}
%	\begin{itemize}
%		\item Euclid of Alexandria, \textit{Optics}: 300 BC, geometry of vision
%		\item Hero of Alexandria, \textit{Catoptrica}: 1st c
%		\begin{itemize}
%			\item geometrical optics of mirror (catoptics)
%			\item emission theory: \textit{visual rays procced at great speed from the eye to the object and are reflected}
%		\end{itemize}
%		\item Claudius Ptolemy, \textit{Optics}: 2nd c, geometrical optics of reflection and refraction
%		\item Ibn Sahl: 10th c Persia
%		\begin{itemize}
%			\item compiled Ptlolemy \textit{Optics}
%			\item law refraction, equivalent to Snell's law
%		\end{itemize}
%		\item Ibn al-Haytham (Alhazen): 11th c, \textit{Book of Optics}
%		\begin{itemize}
%			\item vision occurs because of rays entering the eye (against emission theory)
%			\item pinhole, concave lenses, magnifying glasses
%			\item 'the father of modern optics'
%		\end{itemize}
%		\item Robert Grosseteste: 12th/13th c. English bishop
%		\item Roger Bacon: 13th c. English Franciscan
%		\begin{itemize}
%			\item corrective lenses
%		\end{itemize}
%	\end{itemize}
%\end{frame}
%
%
%\begin{frame}[allowframebreaks]{development of early modern optics}
%	\begin{itemize}
%		\item Johannes Kepler (1571-1630): \textit{Astronomiae Pars Optica}
%		\begin{itemize}
%			\item inverse square law of intensity of light
%			\item reflection by flat and curved mirrors
%			\item pinhole camera
%			\item astronomical implications of parallax
%			\item apparent sizes of heavenly bodies	
%		\end{itemize}
%		\item Willebrord Snellius (1580-1626)
%		\begin{itemize}
%			\item law of refraction: Snell's law
%		\end{itemize}
%		\item Ren\`e Descartes (1596-1650)
%		\item Christiaan Huygens (1629-1695)
%		\item Isaac Newton (1643-1727): \textit{Optiks}
%		\begin{itemize}
%			\item refraction, prism, spectrum
%			\item Newton's theory of color, chromatic aberration
%			\item reflecting/Newtonian telescope
%			\item particle/corpuscular theory of light
%		\end{itemize}
%		\item Francesco Maria Grimaldi (1618-1663): diffraction
%		\item Thomas Young (1773-1829)
%		\begin{itemize}
%			\item double-slit interferometer
%		\end{itemize}
%		\item Augustin-Jean Fresnel (1788-1827): France
%		\begin{itemize}
%			\item wave theory of light
%			\item Fresnel lens: catadioptric (reflective/refractive)
%		\end{itemize}
%	\end{itemize}
%\end{frame}
%
%
%\begin{frame}[allowframebreaks]{historical optical instruments}
%	\begin{itemize}
%		\item lenses
%		\item eyeglasses
%		\begin{itemize}
%			\item ca 1286 Pisa, Italy: first pair of eyeglasses 
%		\end{itemize}
%		\item refracting telescope
%		\begin{itemize}
%			\item refracting telescopes existed since early 1600s, Netherlands
%			\item improvement by Gelileo
%		\end{itemize}
%		\item reflecting telescope
%		\begin{itemize}
%			\item Newton's study of chromatic aberration
%			\item 1671 Newtonian telescope
%		\end{itemize}
%		\item compound microscope
%		\begin{itemize}
%			\item 1620 first known compound microscope, Netherlands
%		\end{itemize}
%	\end{itemize}
%\end{frame}
%
%
%\begin{frame}[allowframebreaks]{quantum optics}
%\end{frame}
%
%
%\section{imaging: catching the light}
%
%\begin{frame}{imaging science}
%	\begin{itemize}
%		\item imaging modes 
%		\item imaging issues
%	\end{itemize}
%	\vspace{5 mm}
%	$\Rightarrow$ \href{https://www.pnas.org/doi/pdf/10.1073/pnas.90.21.9746}{RN Beck, "Overview of imaging science," PNAS 90 9746-9750 (1993)} 
%\end{frame}
%
%
%\begin{frame}[allowframebreaks]{imaging modes}
%	\begin{itemize}
%		\item microscopy
%		\begin{itemize}
%			\item light; UV; X-ray; fluorescence
%			\item electron and ion beam
%			\item ultrasound and acoustic
%			\item tunneling; atomic force; magnetic force
%		\end{itemize}
%		\item x-ray and CAT
%		\item magnetic resonance imaging
%		\item PET
%		\item ultrasound
%		\item holography and pseudo-holography
%		\item telescopic imaging
%		\begin{itemize}
%			\item x-ray, UV, optical, IR, radio
%		\end{itemize}
%		\item computer generated
%		\item graphic arts and sciences
%		\begin{itemize}
%			\item drawing, painting, lithography, printing
%			\item photography, cinematography, television, video
%		\end{itemize}
%		\item computer vision
%		\begin{itemize}
%			\item ccd, cmos
%		\end{itemize}
%		\item human vision
%	\end{itemize}
%\end{frame}
%
%
%\begin{frame}[allowframebreaks]{imaging issues}
%	\begin{itemize}
%		\item objects to be imaged
%		\begin{itemize}
%			\item physical, chemical, isotopic composition in space and time
%			\item emissivity, reflectivity, fluorescence, transparency, opacity, scattering properties
%			\item relaxation times, conductivity, force fields
%		\end{itemize}
%		\item image-data acquisition
%		\begin{itemize}
%			\item detection and localization
%			\item angular, spatial, temporal
%		\end{itemize}
%		\item image recovery
%		\begin{itemize}
%			\item reconstruction and processing
%			\item analytic, statistical, probabilistic methods
%			\item linear \& non-linear, stationary \& non-stationary methods
%		\end{itemize}
%		\item image recording and distribution
%		\begin{itemize}
%			\item compression, storage, retrieval, transmission, networking
%		\end{itemize}
%		\item image display and visualization
%		\begin{itemize}
%			\item monochrome vs color
%			\item 2D vs 3D
%			\item static vs dynamic
%			\item single mode vs fused
%		\end{itemize}
%		\item image observation
%		\item image analysis
%		\begin{itemize}
%			\item segmentation, measurement, morphologic analysis
%			\item pattern recognition, feature extraction
%			\item expert systems, artificial intelligence
%		\end{itemize}
%		\item image evaluation
%		\begin{itemize}
%			\item measure of image quality, SNR, information content, object/image correspondence
%		\end{itemize}
%	\end{itemize}
%\end{frame}
%
%
%\section{holography: historical outline}
%
%\begin{frame}{analog holography}
%	\begin{itemize}
%		\item W Lawrence Bragg 
%		\item Dennis Gabor 
%		\item x-ray diffraction microscopy
%		\item invention of laser
%		\item Emmett Leith \& juris Upatnieks 
%		\item Yuri N Denisyuk 
%		\item Stephen Benton 
%	\end{itemize}
%\end{frame}
%
%
%\begin{frame}{digital holography}
%	\begin{itemize}
%		\item Joseph Goodman
%		\item Ulf Schnars \& Werner Jueptner
%		\item Christian Depeursinge
%		\item Pietro Ferraro
%		\item Ichiru Yamaguchi
%		\item Barun Javidi
%	\end{itemize}
%\end{frame}
%
%
%\begin{frame}{computer geneerated holography}
%	\begin{itemize}
%		\item AW Lohmann
%		\item Leonid Yaroslavsky
%	\end{itemize}
%\end{frame}
%
%
%\section{fourier optics}
%
%\begin{frame}{Discrete Fourier transform}
%
%\end{frame}
%
%\section{holography: basic principles and processes}
%
%
%
%\section{digital holography: numerical diffraction}
%
%
%
%\section{development of digital holography}
%
%
%
%\section{kimlab holography highlights}
%
%
%
%\section{new developments in digital holography}
%
%
%
%\section{applications of digital holography}
%
%
%
%\section{commercialization of digital holography}
%
%
%\frame{}
%\end{document}
%

%\section{holography: historical outline}
%
%\begin{frame}{holography: historical outline}
%\begin{itemize}
%\item W Lawrence Bragg (1890-1971)
%\item Dennis Gabor (1900-1979)
%\item x-ray diffraction microscopy
%\item invention of laser
%\item Emmett Leith (1927-2005)
%\item juris upatnieks (1936-)
%\item Yuri N Denisyuk (1927-2006)
%\item Stephen Benton (1941-2003)
%\vspace{5mm}
%\item Joseph Goodman 
%\end{itemize}
%\end{frame}
%
%
%
%\begin{frame}
%
%\end{frame}
%
%
%\end{document}
%
