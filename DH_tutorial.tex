\documentclass[t, aspectratio=169]{beamer}
\usepackage[utf8]{inputenc}

\usetheme[progressbar=frametitle]{Madrid}
\usecolortheme{crane}
\setbeamertemplate{frame numbering}[fraction]

\title{Multi-Wavelength Techniques in Digital Holography: Tutorial}
\author{Myung K Kim}
\institute{Dept of Physics, University of South Florida, Tampa, FL USA 33620}
\date{August 2022}


\begin{document}

\begin{frame}
	Optica: Digital Holography and Three-Dimensional Imaging\\
	Cambridge University, Cambridge, UK
	\titlepage
\end{frame}


\begin{frame}{abstract}
Basic principles and techniques will be described for multi-wavelength or multi-illumination-angle methods for extending capabilities of digital holography for surface profilomentry by optical phase unwrapping and topography/tomography by snthesized optical coherence.
\end{frame}


\begin{frame}[allowframebreaks]{Table of Contents}
	\tableofcontents[hideallsubsections]
\end{frame}


\section{holography background}

\begin{frame}{\secname}
	\begin{itemize}
		\item holography cartoon
		\item basic theoretical description of holography
		\item highlights in history of analog holography 
		\item digital holography cartoon
		\item highlights in history of digital holography
	\end{itemize}
\end{frame}


\section{digital holography system}

\begin{frame}{\secname}
	\begin{itemize}
		\item theory of digital holography
		\item interferometer configurations
		\item apparatus and hardware
		\item software for experimental control
		\item software for numerical calculation
	\end{itemize}
\end{frame}


\section{multi-wavelength background}

\begin{frame}{\secname}
	\begin{itemize}
		\item cartoon: two-wavelength interference
		\item multi-wavelength interferometric methods: historical examples
		\item multi-wavelength analog holographic interference: examples
	\end{itemize}
\end{frame}


\section{two-wavelength DHQPM}

\begin{frame}{\secname}
	\begin{itemize}
		\item theory and simulation of two-wavelength interference
		\item two-wavelength DHQPM: basic expts
		\item two-wavelength holography and interferography: examples
		\item methods of multi-wavelength generation
	\end{itemize}
\end{frame}


\section{multi-wavelength DHQPM}

\begin{frame}{\secname}
	\begin{itemize}
		\item theory and simulation of multi-wavelength QPM
		\item multi-wavelength QPM: basic expts
		\item multi-wavelength QPM: examples		
	\end{itemize}
\end{frame}


\section{wavelength scanning DIH}

\begin{frame}{\secname}
	\begin{itemize}
		\item theory of wavelength scanning DIH
		\item wavelength scanning DIH: basic expts
		\item wavelength scanning DIH: examples
		\item comparison with full field OCT
	\end{itemize}
\end{frame}


\section{methods of multi-wavelength generagtion}

\begin{frame}{\secname}
	\begin{itemize}
		\item multiple lasers
		\item tunable laser
		\item refractive index
		\item angle tuning
	\end{itemize}
\end{frame}


\section{multi-angle DHQPM - gammascan}

\begin{frame}{\secname}
	\begin{itemize}
		\item gamma scan: basic expts
		\item gamma scan: examples
	\end{itemize}
\end{frame}


\section{angle scanning DIH - alphascan}

\begin{frame}{\secname}
	\begin{itemize}
		\item alpha scan: basic expts
		\item alpha scan: examples
	\end{itemize}
\end{frame}


\section{multi-wavlength/angle holography system}

\begin{frame}{\secname}
	\begin{itemize}
		\item optomechanical system
		\item software for holographic data acquisition
		\item software for numerical calculation
	\end{itemize}
\end{frame}


\section{further development}

\begin{frame}{\secname}
	\begin{itemize}
		\item current state of the art
		\item current issues
		\item competing trending technologies
		\item potential solutions and advantages
	\end{itemize}
\end{frame}


\section{references}


\frame{}


\begin{frame}{todolist}
	\begin{itemize}
		\item list of primary papers on multi-wavelength methods
		\item search secondary papers and recent development
		\item start bibliography
		\item start holography background
		\item start first Inkspace/Gimp figures
		\item add slides ...
	\end{itemize}
\end{frame}


\end{document}





