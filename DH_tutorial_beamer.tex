\documentclass[t, aspectratio=169]{beamer}
\usepackage[utf8]{inputenc}
\usepackage{amsmath}

\usetheme[progressbar=frametitle]{Madrid}
\usecolortheme{crane}
\setbeamertemplate{frame numbering}[fraction]

\usefonttheme[onlymath]{serif}
\usepackage{multicol}
\usepackage[absolute,overlay]{textpos}

\usepackage{bibentry}

%\title{Multi-Wavelength Techniques in Digital Holography: Tutorial}
%\author{Myung K Kim}
%\institute{Dept of Physics, University of South Florida, Tampa, FL USA 33620}
%\date{August 2022}

\title{Multi-Wavelength Techniques in Digital Holography}
\author{Myung K Kim}
\institute{University of South Florida}
\date{August 2022}


\begin{document}


\begin{frame}
	Optica: Digital Holography and Three-Dimensional Imaging\\
	Cambridge University, Cambridge, UK
	\titlepage
\end{frame}


\begin{frame}{abstract}
	\vspace{10 mm}
	\centering
Basic principles and techniques will be described for multi-wavelength or multi-illumination-angle mehtods for extending capabilities of digital holography for surface profilometry by optical phase unwrapping and topography/tomography by synthesized optical coherence
\end{frame}


\begin{frame}{acknowledgement}
	\begin{multicols}{2}
		\begin{itemize}
			\item{James Gass}
			\item{Aaron Dakof}
			\item{Dan Parshall}
			\item{Lingfeng Yu}
			\item{Chris Mann}
			\item{Alejandro Rostrepo}
			\item{Nilanthi Warnasooriya}
			\item{Alex Khmaladze}
			\item{Mariana Potcoava}
			\item{Leo Krzewina}
			\item{Bill Ash}
			\item{David Clark}
			\item{Xiao Yu}
			\item{Changgeng Liu}
			\item{Mauricio Flores}
			\item{Jisoo Hong}
			\item{Jiawen Weng}
			\item{Changwon Jang}
			\vspace{5 mm}
			\item{National Science Foundation}
			\item{National Eye Institute}
			\item{HoloVision}
			\item{HCube Optics}
		\end{itemize}
	\end{multicols}
\end{frame}


\begin{frame}[allowframebreaks]{Table of Contents}
	\tableofcontents[hideallsubsections]
\end{frame}


\section{digital holography}
\begin{frame}[c]
	\centering\LARGE\textbf{\secname}
\end{frame}


\begin{frame}{Denis Gabor: Microscopy by reconstructed wave-fronts}
	\vspace{-3 mm}
	\small Proc. Royal Soc London 197, 454 (1949)
	\begin{columns}
	    \begin{column}{.6\textwidth}
			\begin{figure}
				\includegraphics[height=55 mm]{DH_tutorial_md.assets/image-20220426134139308}
			\end{figure}
		\end{column}
		\begin{column}{.4\textwidth}
			\begin{itemize}
				\item aberrations in electron microscope
				\item optical demonstration
			\end{itemize}
		\end{column}
	\end{columns}
\end{frame}


\begin{frame}{EN Leigth \& J Upatnieks: Wavefront Reconstruction with Diffused Illumination and Three-Dimensional Objects}
	\vspace{-3 mm}
	\small J Opt Soc Am, 54, 1295(1964)
	\begin{columns}
		\begin{column}{.6\textwidth}
			\begin{figure}
				\includegraphics[height=55 mm]{DH_tutorial_md.assets/image-20220426135144501}
			\end{figure}
		\end{column}
		\begin{column}{.4\textwidth}
			\begin{itemize}
				\item laser: coherent illumination
				\item full 3D recording and reconstruction
			\end{itemize}
		\end{column}
	\end{columns}
\end{frame}


\begin{frame}{aaa}
	\vspace{-3 mm}
	\small aaa
	\begin{columns}
		\begin{column}{.6\textwidth}
			\begin{figure}
				\includegraphics[height=55 mm]{DH_tutorial_md.assets/image-20220426135144501}
			\end{figure}
		\end{column}
		\begin{column}{.4\textwidth}
			\begin{itemize}
				\item aaa
			\end{itemize}
		\end{column}
	\end{columns}
\end{frame}


\begin{frame}{JW Goodman: Digital image formation from electronically detected holograms}
	\vspace{-3 mm}
	\small Appl Phys Lett 11, 77 (1967)
	\begin{columns}
		\begin{column}{.6\textwidth}
			\begin{figure}
				\includegraphics[height=55 mm]{DH_tutorial_md.assets/image-20220426135625638}
			\end{figure}
		\end{column}
		\begin{column}{.4\textwidth}
			\begin{itemize}
				\item aaa
			\end{itemize}
		\end{column}
	\end{columns}
\end{frame}


\begin{frame}{U Schnars \& W Jueptner: Direct recording of holograms by a CCD target and numerical reconstruction}
	\vspace{-3 mm}
	\small Opt Lett 24, 291 (1999)
	\begin{columns}
		\begin{column}{.6\textwidth}
			\begin{figure}
				\includegraphics[height=55 mm]{DH_tutorial_md.assets/image-20220426140019172}
			\end{figure}
		\end{column}
		\begin{column}{.4\textwidth}
			\begin{itemize}
				\item aaa
			\end{itemize}
		\end{column}
	\end{columns}
\end{frame}


\begin{frame}{C Depeursinge: Digital holography for quantitative phase-contrast imaging}
	\vspace{-3 mm}
	\small Opt Lett 24, 291 (1999)
	\begin{columns}
		\begin{column}{.6\textwidth}
			\begin{figure}
				\includegraphics[height=55 mm]{DH_tutorial_md.assets/image-20220426140050211}
			\end{figure}
		\end{column}
		\begin{column}{.4\textwidth}
			\begin{itemize}
				\item aaa
			\end{itemize}
		\end{column}
	\end{columns}
\end{frame}


\begin{frame}{holographic terms}
	\begin{figure}
		\includegraphics[width=4.0 in]{DH_tutorial_md.assets/image-20220415152931893}
	\end{figure}	
hologram recording \\
	\begin{center}
		$I = |E_R + E_O|^2 = |E_R|^2 + |E_O|^2 +E_R^*E_O + E_RE_O^*$
	\end{center}
	\pause
hologram reconstruction \\
	\begin{center}
		$E'_RI = E'_R|E_R+E_O|^2 
		=E'_R|E_R|^2 + E'_R|E_O|^2 + E'_RE_R^*E_O + E'_RE_RE_O^*$
	\end{center}
\end{frame}


\begin{frame}{holography of plane waves}
	\begin{figure}
		\includegraphics[width=3.0 in]{DH_tutorial_md.assets/image-20220415152849132}
	\end{figure}
\end{frame}


\begin{frame}{holography of spherical waves}
	\begin{figure}
		\includegraphics[width=3.0 in]{DH_tutorial_md.assets/image-20220415152849132}
	\end{figure}
\end{frame}


\begin{frame}{diffraction from a 2D aperture}
	\begin{figure}
		\includegraphics[height=1.5 in]{DH_tutorial_md.assets/image-20220415162604622}
	\end{figure}
	\begin{gather*}
E(x,y;z) = -\frac{ik}{2\pi}\iint_{\Sigma_0} dx_0dy_0\,E_0(x_0,y_0)\frac{\exp(ikr)}{r}
	\end{gather*}
\end{frame}


\begin{frame}{Huygens convoluton method}	
	\begin{gather*}	
E(x,y;z) \approx -\frac{ik}{2\pi z} \iint_{\Sigma_0} dx_0dy_0\,E_0(x_0,y_0)\exp\left[ik\sqrt{(x-x_0)^2 + (y-y_0)^2 + z^2 }\right] \\
= E_0 \odot S_H 
	\end{gather*}
	\pause
	\begin{gather*}
\textbf{spherical wavefront} \qquad S_H(x,y;z) = - \frac{ik}{2\pi z} \exp\left[ik\sqrt{x^2+y^2+z^2}\right]
	\end{gather*}
\end{frame}


\begin{frame}{Fresnel transform method}	
	\vspace{-5 mm}
	\begin{gather*}	
r \approx z + \frac{x^2+y^2}{2z} \\
	\end{gather*}
	\pause
	\vspace{-8 mm}
	\begin{gather*}
E(x,y) = - \frac{ik}{2\pi z} \exp\left[ikz + \frac{ik}{2z} (x^2+y^2)\right] \iint dx_0\,dy_0E_0(x_0,y_0)\exp\left[\frac{ik}{2z}(x_0^2+y_0^2)\right] \\
\hspace{50mm}\times\exp\left[-\frac{ik}{z}(xx_0+yy_0)\right] \\ 
=\exp\left[\frac{ik}{2z}(x^2+y^2)\right]\mathfrak F\{E_0S_F\} \\
	\end{gather*}
	\pause
	\vspace{-8 mm}
	\begin{gather*}
\textbf{parabolic wave front} \qquad S_F(x,y;z) = -\frac{ik}{2\pi z}\exp\left[ikz + \frac{ik}{2z}(x^2+y^2)\right]
	\end{gather*}
\end{frame}


%\begin{frame}{propagation of angular spectrum}
%	\begin{textblock*}{110mm}(5mm,20mm)
%		\textbf{angular spectrum of input wave}: plane wave components
%		\begin{gather*}
%A_0(k_x,k_y) = \mathfrak F\left\{E_0(x_0,y_0)\right\}\left[k_x,k_y\right] \\
%=\frac{1}{2\pi}\iint_{\Sigma_0} dx_0 dy_0 E_0(x_0,y_0) \exp\left[-i(k_x x_0 + k_y y_0)\right] 
%		\end{gather*} 
%		output wave is the sum of propagated input plane wave components
%		\begin{gather*}
%E(x,y,z) = \mathfrak F^{-1}\left\{A_0(k_x,k_y) \exp[ik_z z]\right\}[x,y] \\
%= \mathfrak F^{-1}\left\{\mathfrak F\{E_0\}\exp\left[i\sqrt{k^2 - k_x^2 - k_y^2}\ z\right]\right\} \\
%k_z = \sqrt{k^2 - k_x^2 - k_y^2} \qquad \left[ k_x^2 + k_y^2 \le k^2 \right]	
%		\end{gather*}
%	\end{textblock*}
%	\begin{textblock*}{12cm}(7cm,15mm) % {block width} (coords)
%		\begin{figure}
%			\includegraphics[width=1.75 in]{DH_tutorial_md.assets/image-20220415153149971}
%		\end{figure}
%	\end{textblock*}
%\end{frame}


\begin{frame}{propagation of angular spectrum}
	\begin{figure}
		\includegraphics[height=1.25 in]{DH_tutorial_md.assets/image-20220415153149971}
	\end{figure}
	\textbf{angular spectrum of input wave}: plane wave components
	\begin{gather*}
A_0(k_x,k_y) = \mathfrak F\left\{E_0(x_0,y_0)\right\}\left[k_x,k_y\right] \\
=\frac{1}{2\pi}\iint_{\Sigma_0} dx_0 dy_0 E_0(x_0,y_0) \exp\left[-i(k_x x_0 + k_y y_0)\right]
	\end{gather*}
\end{frame}


\begin{frame}{propagation of angular spectrum}
	\begin{figure}
		\includegraphics[height=1.25 in]{DH_tutorial_md.assets/image-20220415153149971}
	\end{figure}
output wave is the sum of propagated input plane wave components
	\begin{gather*}
E(x,y,z) = \mathfrak F^{-1}\left\{A_0(k_x,k_y) \exp[ik_z z]\right\}[x,y] \\
= \mathfrak F^{-1}\left\{\mathfrak F\{E_0\}\exp\left[i\sqrt{k^2 - k_x^2 - k_y^2}\ z\right]\right\} \\
k_z = \sqrt{k^2 - k_x^2 - k_y^2} \qquad \left[ k_x^2 + k_y^2 \le k^2 \right]	
	\end{gather*}
\end{frame}


\begin{frame}{FFT: fast Fourier transform}
	\begin{figure}
		\includegraphics[height=1.15 in]{DH_tutorial_md.assets/image-20220416111943805}
	\end{figure}
	\begin{gather*}
\mathfrak F\left\{f(x,y)\right\}[k_x,k_y] = \frac{1}{2\pi}\iint dx\,dy\,f(x,y)\exp\left[-i(k_x x + k_yy)\right] = F(k_x,k_y) \\
\mathfrak F^{-1}\left\{F(k_x,k_y)\right\}[x,y] = \frac{1}{2\pi}\iint dk_x\,dk_y\,F(k_x,k_y)\exp\left[+i(k_x x + k_yy)\right] = f(x,y) \\
K_x = 2\pi/dx \qquad dk_x = 2\pi/A_x = K_x/N_x \\
K_y = 2\pi/dy \qquad dk_y = 2\pi/A_y = K_y/N_y
	\end{gather*}
\end{frame}


\begin{frame}{range of recordable wavefront slope}
	\begin{itemize}
		\item $K_x/k=\lambda/dx$: maximum slope of wavefront before violating Nyquist
		\item $dk/k = \lambda/A_x$: minimum slope of wavefront to record one cycle over the frame size $A_x$ 
	\end{itemize}
\end{frame}


\begin{frame}{HCM vs ASM: propagation of spherical \& plane waves}
	\begin{figure}
		\includegraphics[height=2.0 in]{DH_tutorial_md.assets/image-20220415160427601}
	\end{figure}
\end{frame}


\begin{frame}{FTM vs HCM vs ASM}
	\begin{figure}
		\includegraphics[height=2.0 in]{DH_tutorial_md.assets/image-20220415160459990}
	\end{figure}
	\textbf{FTM}: 
	\[ Z_{\min} = \frac{X_0^2}{N\lambda}; \qquad\qquad Z_{\max} = \frac{X_0^2}{2\lambda} \]
\end{frame}


\begin{frame}[fragile]{sample Python code for AS diffraction}
	\begin{semiverbatim}
def asdiffract(EE):    
    xx, yy = meshgrid(-ax/2:dx:ax/2-dx, -ay/2:dy:ay/2-dy)
    k = 2*pi/wlen
    dkx, dky = 2*pi/ax, 2*pi/ay
    akx, aky = 2*pi/dx, 2*pi/dy
    kxx, kyy = meshgrid(-akx/2:dkx:akx/2-dkx, -aky/2:dky:aky/2-dky)
    
    FF = fftshift(fft(EE))    
    GG = exp(1j*z*sqrt(k**2 - kxx**2 - kyy**2))    
    HH = ifft(ifftshift(FF * GG))
    
    return HH    
	\end{semiverbatim}
\end{frame}


\section{digital holography system}
\begin{frame}[c]
	\centering\LARGE\textbf{\secname}
\end{frame}

\begin{frame}{DHM apparatus - reflection}
	\begin{figure}
		\includegraphics[height=2.5 in]{DH_tutorial_md.assets/image-20220427220207929}
	\end{figure}
\end{frame}


\begin{frame}{DHM apparatus - transmission}
	\begin{figure}
		\includegraphics[height=1.75 in]{DH_tutorial_md.assets/image-20220416171658927}
	\end{figure}
\end{frame}


\begin{frame}{Gabor holography}
	\begin{figure}
		\includegraphics[width=120mm]{DH_tutorial_md.assets/image-20220426154854457}
	\end{figure}
\end{frame}


\begin{frame}{in-line holography}
	\begin{figure}
		\includegraphics[width=120mm]{DH_tutorial_md.assets/image-20220426154916787}
	\end{figure}
\end{frame}


\begin{frame}{phase-shifting digital holography}
	\begin{figure}
		\includegraphics[width=80mm]{DH_tutorial_md.assets/image-20220426155721421}
	\end{figure}
\end{frame}


\begin{frame}{phase-shifting digital holography}
	\begin{gather*}
E_O(x,y) = \mathcal{E}_O(x,y)\exp[i\Phi(x,y)] \\
E_n = \mathcal{E}_R \exp(i2\pi n/N) \\
\alpha_n = \frac{2\pi}{N}n \qquad n=1,2,3,\cdots,N \\
I_n = |E_R + E_O|^2 = \mathcal{E}_R^2 + \mathcal{E}_O^2 + \mathcal{E}_R\mathcal{E}_O e^{i(\alpha_n-\Phi)} + \mathcal{E}_R\mathcal{E}_O e^{i(-\alpha_n+\Phi)} \\
S=\frac{1}{N}\sum_{n=1}^N I_n e^{i\alpha_n} = \frac{1}{N}\left\{\left[\mathcal{E}_R^2 + \mathcal{E}_O^2 \right]\sum_n e^{i\alpha_n} + \mathcal{E}_R\mathcal{E}_O e^{-i\Phi}\sum_n e^{2i\alpha_n} + \mathcal{E}_R\mathcal{E}_O e^{i\Phi}N \right\} \\
= \mathcal{E}_R\mathcal{E}_O(x,y)\exp[i\Phi(x,y)] \\
E_O(x,y) = \frac{1}{N\mathcal E_R}\sum_{n=1}^N I_n \exp(i2\pi n/N)
	\end{gather*}
\end{frame}


\begin{frame}{phase-shifting digital holography}
\[ \textrm{for} N=4:\hspace{10 mm} \Phi(x,y) = \tan\frac{I_1 - I_3}{I_2 - I_4} \]	
	\begin{figure}
		\includegraphics[width=95 mm]{DH_tutorial_md.assets/image-20220426155750402}
	\end{figure}	
\end{frame}


\begin{frame}[fragile]{sample Python code for PSDH}
	\begin{semiverbatim}
def psdh(HHH):
	ny, nx, nph = shape(HHH)
	EE = zeros(ny,nx)
	for n in range(nph):
		EE += HH[:,:,n] * exp(1j*n*2*pi/nph)
    EE = EE/nph
    return EE
	\end{semiverbatim}
\end{frame}


\begin{frame}{off-axis holography}
	\begin{figure}
		\includegraphics[width=120mm]{DH_tutorial_md.assets/image-20220426154947369}
	\end{figure}
\end{frame}


\begin{frame}{spatial frequency bandwidth in off-axis holography}
	\begin{figure}
		\includegraphics[width=40 mm]{DH_tutorial_md.assets/image-20220416170733456}
	\end{figure}
	\begin{gather*}
B = \frac{2}{2+3\sqrt{2}}K = \beta K \qquad \beta=0.32 \\
K_0 = \frac{3}{2\sqrt{2}}B = \frac{3}{2\sqrt{2}+6} K = \gamma K \qquad \gamma=0.34 \qquad \sqrt{2}\,\gamma = 0.48 
	\end{gather*}
\end{frame}


\begin{frame}[fragile]{sample OXDH.py code}
	\begin{semiverbatim}
def oxdh(HH, qq):
    xx, yy = meshgrid(-ax/2:dx:ax/2-dx, -ay/2:dy:ay/2-dy)
    k = 2*pi/wlen
    dkx, dky = 2*pi/ax, 2*pi/ay
    akx, aky = 2*pi/dx, 2*pi/dy
    kxx, kyy = meshgrid(-akx/2:dkx:akx/2-dkx, -aky/2:dky:aky/2-dky)
    qx0, qy0, qx, qy = qq	% angular position & width
    kx0, ky0, kx, ky = k*sin(qx0), k*sin(qy0), k*sin(qx), k*sin(qy)   
    FF = fftshift(fft(HH))    
    BB = (((kxx-kx0)/kx)**2 + ((kyy-ky0)/ky)**2) < 1
    FF = FF * BB
    EE = ifft(ifftshift(FF)) 
    GG = exp(-1j*(kx0*xx + ky0*yy))
    EE = EE * GG    
    return EE
	\end{semiverbatim}
\end{frame}


\begin{frame}{CJ Mann, et al.: High-resolution quantitative phase-contrast microscopy by digital holography}
	\vspace{-3 mm}
	\small Opt Expr 13, 8693 (2005)
	\begin{columns}
		\begin{column}{.6\textwidth}
			\begin{figure}
				\includegraphics[height=55 mm]{DH_tutorial_md.assets/image-20220416215324683}
			\end{figure}
		\end{column}
		\begin{column}{.4\textwidth}
			\begin{itemize}
				\item aaa
			\end{itemize}
		\end{column}
	\end{columns}
\end{frame}


%\begin{frame}{DHM example}
%Mann, C. J., Yu, L. F., Lo, C. M., \& Kim, M. K. (2005). High-resolution quantitative phase-contrast microscopy by digital holography. *OPTICS EXPRESS*, *13*(22), 8693–8698
%	\begin{figure}
%		\includegraphics[width=90 mm]{DH_tutorial_md.assets/image-20220416215324683}
%	\end{figure}
%\end{frame}


\begin{frame}{Depeursinge, et al.: Spatial analysis of erythrocyte membrane fluctuations by digital holographic microscopy}
	\vspace{-3 mm}
	\small Blood Cells, Molecules, and Diseases 42, 228–232 (2009)
	\begin{columns}
		\begin{column}{.6\textwidth}
			\begin{figure}
				\includegraphics[height=55 mm]{DH_tutorial_md.assets/image-20220427092446148}
			\end{figure}
		\end{column}
		\begin{column}{.4\textwidth}
			\begin{itemize}
				\item aaa
			\end{itemize}
		\end{column}
	\end{columns}
\end{frame}


%\begin{frame}{DHM example}
%Rappaz, B., Barbul, A., Hoffmann, A., Boss, D., Korenstein, R., Depeursinge, C., Magistretti, P. J., \& Marquet, P. (2009). Spatial analysis of erythrocyte membrane fluctuations by digital holographic microscopy. *Blood Cells, Molecules, and Diseases*, *42*(3), 228–232
%	\begin{figure}
%		\includegraphics[width=100 mm]{DH_tutorial_md.assets/image-20220427092446148}
%	\end{figure}
%\end{frame}


\begin{frame}{Time-lapse of unstained, dividing and migrating cells}
	\vspace{-3 mm}
	\small wikipedia
	\begin{columns}
		\begin{column}{.6\textwidth}
			\begin{figure}
				\includegraphics[height=55 mm]{DH_tutorial_md.assets/DHM_cell_wiki.png}
			\end{figure}
		\end{column}
		\begin{column}{.4\textwidth}
			\begin{itemize}
				\item aaa
			\end{itemize}
		\end{column}
	\end{columns}
\end{frame}


%\begin{frame}{DHM example}
%Time-lapse of unstained, dividing and migrating cells
%	\begin{figure}
%		\includegraphics[width=75 mm]{DH_tutorial_md.assets/DHM_cell_wiki.png}
%	\end{figure}
%\end{frame}


\section{two-wavelength optical phase unwrapping}
\begin{frame}[c]
	\centering\LARGE\textbf{\secname}
\end{frame}


\begin{frame}{multi-wavelength DH: motivation}
multi-wavelength: motivation
	\begin{itemize}
		\item QPM with height/thickness range beyond wavelength
	\end{itemize}	
multi-wavelength: history
	\begin{itemize}
		\item HILDEBRAND, B. P., \& HAINES, K. A. (1967). MULTIPLE-WAVELENGTH AND MULTIPLE-SOURCE HOLOGRAPHY APPLIED TO CONTOUR GENERATION. *JOURNAL OF THE OPTICAL SOCIETY OF AMERICA*, *57*(2), 155+. https://doi.org/10.1364/JOSA.57.000155
		\item Cheng, Y.-Y., \& Wyant, J. C. (1984). Two-wavelength phase shifting interferometry. *Appl. Opt.*, *23*(24), 4539–4543. https://doi.org/10.1364/AO.23.004539
	\end{itemize}
wavelength-scanning: history
\end{frame}


\begin{frame}{QPM with single wavelength}
	\begin{gather*}
E(x,y) = a(x,y)\cdot\exp[i\Phi(x,y)] \\
a \ //\ b \quad \equiv \textrm{mod}\left(a+\frac{b}{2},\ b\right)-\frac{b}{2} \qquad \in \left[-\frac{b}{2}, \frac{b}{2}\right]\\
\Phi(x,y) = \left[ 2\pi\ \frac{Z(x,y)}{\lambda} \right] \ //\ 2\pi \qquad \in [-\pi,\pi] \\
Z_\lambda(x,y) = \lambda\ \frac{\Phi(x,y)}{2\pi} \qquad \in \left[-\frac{\lambda}{2}, \frac{\lambda}{2}\right]
	\end{gather*}
\end{frame}


\begin{frame}{2WOPU}
	\begin{gather*}
E_n(x,y) = a_n(x,y)\cdot\exp[i\Phi_n(x,y)] \\
E_{12}(x,y) \equiv E_1\cdot E_2^* = a_{12}\cdot\exp[i\Phi_{12}(x,y)]\\
\Phi_{12}(x,y) \equiv (\Phi_1 - \Phi_2) \ //\ 2\pi = \left[ 2\pi\ \frac{Z(x,y)}{\Lambda_{12}} \right]\ //\ 2\pi \qquad \in [-\pi,\pi] \\
\frac{1}{\Lambda_{12}} = \frac{1}{\lambda_1} - \frac{1}{\lambda_2} \qquad\rightarrow\qquad \Lambda_{12} = -\frac{\lambda_1 \lambda_2}{\lambda_1 - \lambda_2} \\
Z_{12}(x,y) = \Lambda_{12}\cdot\frac{\Phi_{12}(x,y)}{2\pi} \qquad \in\left[-\frac{\Lambda_{12}}{2}, \frac{\Lambda_{12}}{2} \right] \\
\frac{\Lambda}{\lambda} = \frac{\lambda}{\Delta\lambda}
	\end{gather*}
\end{frame}


\begin{frame}{2WOPU process}
	\begin{figure}
		\includegraphics[height=70 mm]{DH_tutorial_md.assets/image-20220427141039405}
	\end{figure}
\end{frame}


\begin{frame}[fragile]{sample 2WOPU.py code}
	\begin{semiverbatim}
aaa
	\end{semiverbatim}
\end{frame}


\begin{frame}{A Khmaladze, et al.: Phase imaging of cells by simultaneous dual-wavelength reflection digital holography}
	\vspace{-3 mm}
	\small Opt Expr 16, 10900 (2008)
	\begin{columns}
		\begin{column}{.6\textwidth}
			\begin{figure}
				\includegraphics[height=55 mm]{DH_tutorial_md.assets/image-20220427141107699}
			\end{figure}
		\end{column}
		\begin{column}{.4\textwidth}
			\begin{itemize}
				\item aaa
			\end{itemize}
		\end{column}
	\end{columns}
\end{frame}


\begin{frame}{A Khmaladze, et al.: Phase imaging of cells by simultaneous dual-wavelength reflection digital holography}
	\vspace{-3 mm}
	\small Opt Expr 16, 10900 (2008)
	\begin{columns}
		\begin{column}{.6\textwidth}
			\begin{figure}
				\includegraphics[height=55 mm]{DH_tutorial_md.assets/image-20220427141311148}
			\end{figure}
		\end{column}
		\begin{column}{.4\textwidth}
			\begin{itemize}
				\item aaa
			\end{itemize}
		\end{column}
	\end{columns}
\end{frame}


%\begin{frame}[allowframebreaks]{2WOPU example}
%Khmaladze, A., Kim, M., \& Lo, C.-M. (2008). Phase imaging of cells by simultaneous dual-wavelength reflection digital holography. *OPTICS EXPRESS*, *16*(15), 10900–10911
%\begin{figure}
%	\includegraphics[width=90 mm]{DH_tutorial_md.assets/image-20220427141107699}
%\end{figure}
%\begin{figure}
%	\includegraphics[width=70 mm]{DH_tutorial_md.assets/image-20220427141311148}
%\end{figure}
%\end{frame}


\begin{frame}{MS Heimbeck, et al.: Terahertz digital holography using angular spectrum and dual wavelength reconstruction methods}
	\vspace{-3 mm}
	\small Opt Expr 19, 9192 (2011)
	\begin{columns}
		\begin{column}{.6\textwidth}
			\begin{figure}
				\includegraphics[height=55 mm]{DH_tutorial_md.assets/image-20220427141355546}
			\end{figure}
		\end{column}
		\begin{column}{.4\textwidth}
			\begin{itemize}
				\item aaa
				\item Fig.  6.  Steps  in  the  reconstruction  of  the  phase  object  using  dual  wavelength  reconstruction: photograph  (a),  holograms  at  680  and  725  GHz  (b-c),  amplitude  reconstructions  (d-e),  phase reconstructions   (f-g),   unwrapped   reconstruction   (h),   cross-section   through   center   of unwrappred reconstruction (i), pseudo 3D perspective of the unwrapped reconstruction (j).
			\end{itemize}
		\end{column}
	\end{columns}
\end{frame}


%\begin{frame}{2WOPU example}
%Heimbeck, M. S., Kim, M. K., Gregory, D. A., \& Everitt, H. O. (2011). Terahertz digital holography using angular spectrum and dual wavelength reconstruction methods. *OPTICS EXPRESS*, *19*(10), 9192–9200 \\
%Fig.  6.  Steps  in  the  reconstruction  of  the  phase  object  using  dual  wavelength  reconstruction: photograph  (a),  holograms  at  680  and  725  GHz  (b-c),  amplitude  reconstructions  (d-e),  phase reconstructions   (f-g),   unwrapped   reconstruction   (h),   cross-section   through   center   of unwrappred reconstruction (i), pseudo 3D perspective of the unwrapped reconstruction (j).
%\begin{figure}
%	\includegraphics[width=65 mm]{DH_tutorial_md.assets/image-20220427141355546}
%\end{figure}
%\end{frame}


%\begin{frame}{aaa}
%	\vspace{-3 mm}
%	\small aaa
%	\begin{columns}
%		\begin{column}{.6\textwidth}
%			\begin{figure}
%				\includegraphics[height=55 mm]{DH_tutorial_md.assets/image-20220426135144501}
%			\end{figure}
%		\end{column}
%		\begin{column}{.4\textwidth}
%			\begin{itemize}
%				\item aaa
%			\end{itemize}
%		\end{column}
%	\end{columns}
%\end{frame}


\begin{frame}[allowframebreaks]{2WOPU example}
Rinehart, M. T., Shaked, N. T., Jenness, N. J., Clark, R. L., \& Wax, A. (2010). Simultaneous two-wavelength transmission quantitative phase microscopy with a color camera. *OPTICS LETTERS*, *35*(15), 2612–2614
	\begin{figure}
		\includegraphics[width=90 mm]{DH_tutorial_md.assets/image-20220427143118397}
	\end{figure}
Fig. 3. (Color online) Microstructure OPD maps and profiles: (a) 532 nm OPD map after quality-map guided unwrapping, 15 um lateral scale bar; (b) 532 nm OPD map after two-wavelength unwrapping, 15 um lateral scale bar; (c) incorrect object height profile, from the dotted line in (a); (d) object height profile from two-wavelength unwrapping, from the dotted line in (b). The quantitative height measurements ob-tained from QPM agree with the SEM images in Fig. 2.
	\begin{figure}
		\includegraphics[width=70 mm]{DH_tutorial_md.assets/image-20220427143143483}
	\end{figure}
\end{frame}


%\begin{frame}{aaa}
%	\vspace{-3 mm}
%	\small aaa
%	\begin{columns}
%		\begin{column}{.6\textwidth}
%			\begin{figure}
%				\includegraphics[height=55 mm]{DH_tutorial_md.assets/image-20220426135144501}
%			\end{figure}
%		\end{column}
%		\begin{column}{.4\textwidth}
%			\begin{itemize}
%				\item aaa
%			\end{itemize}
%		\end{column}
%	\end{columns}
%\end{frame}


\begin{frame}[allowframebreaks]{2WOPU example}
Lee, Y., Ito, Y., Tahara, T., Inoue, J., Xia, P., Awatsuji, Y., Nishio, K., Ura, S., \& Matoba, O. (2014). Single-shot dual-wavelength phase unwrapping in parallel phase-shifting digital holography. *OPTICS LETTERS*, *39*(8), 2374–2377 \\
Space bandwidths. Proposed method using (a) parallel two-step phase-shifting interferometry and (b) angular multi-plexing technique.
	\begin{figure}
		\includegraphics[width=70 mm]{DH_tutorial_md.assets/image-20220427144722747}
	\end{figure}
	\begin{figure}
		\includegraphics[width=70 mm]{DH_tutorial_md.assets/image-20220427144827974}
	\end{figure}
\end{frame}


\section{h-MWOPU: hierarchical multi-wavelength optical phase unwrapping}
\begin{frame}[c]
	\centering\LARGE\textbf{\secname}
\end{frame}


\begin{frame}{noise in 2WOPU}
	\begin{gather*}
\delta\Phi_n = 2\pi\epsilon \\
\delta Z_n = \lambda_n\ \frac{\delta\Phi_n}{2\pi} = \lambda_n\epsilon \\
\delta\Phi_{12} = \sqrt{2}\ 2\pi\ \epsilon \\
\delta Z_{12} = \sqrt{2}\ \Lambda_{12}\ \epsilon \\
\frac{\delta Z_{12}}{\delta Z_n} = \frac{\sqrt{2}\ \Lambda_{12}}{\lambda_n}
	\end{gather*}
\end{frame}


\begin{frame}[allowframebreaks]{noise reduction by 3WOPU}
	\begin{gather*}
Z_{12\_3}(x,y) = \textrm{round}\left[\frac{Z_{12}(x,y)}{\Lambda_{13}}\right] \Lambda_{13} + Z_{13}(x,y) \\
Z_{12\_3} + \delta Z_{12\_3} = \textrm{round}\left[\frac{Z_{12} + \delta Z_{12}}{\Lambda_{13}} \right] \Lambda_{13} + (Z_{13} + \delta Z_{13}) \\
\delta Z_{12\_3} = \Delta\cdot \Lambda_{13} + \delta Z_{13} = \Delta\cdot\Lambda_{13} + \sqrt{2}\ \Lambda_{13}\ \epsilon \\
	\end{gather*}
$\Delta\cdot\Lambda_{13}$: 		spikes of height $\Lambda_{13}$ scattered near $\textrm{round()}$ boundaries within $\delta Z_{12}$ \\
	\begin{itemize}
		\item require
		\begin{gather*}
\frac{\delta Z_{12}}{\Lambda_{13}} = \sqrt{2}\ \epsilon\ \frac{\Lambda_{12}}{\Lambda_{13}} \ll 1 \\
\alpha = \frac{\Lambda_{13}}{\Lambda_{12}} \gg \sqrt{2}\ \epsilon \\
		\end{gather*}
		\item despiking
		\begin{gather*}
Z_{12\_3} - Z_{12} = s\ \Delta Z \\
\bar{Z}_{12\_3} = Z_{12\_3} - s\ \Lambda_{13}\cdot\left(\Delta Z \gg \delta Z_{13}\right)\  \\
\delta\bar{Z}_{12\_3} = \delta Z_{13} = \sqrt{2}\ \Lambda_{13}\ \epsilon
		\end{gather*}
	\end{itemize}
\end{frame}


\begin{frame}[allowframebreaks]{hierarchical MWOPU}
	\begin{gather*}
\{\lambda_n\} = \lambda_1,\ \lambda_2,\ \lambda_3,\ \cdots \\
E_n(x,y) = a_n(x,y)\cdot\exp[i\Phi_n(x,y)] \\
n \ge 2:\quad E_{1n}(x,y) = E_1\cdot E_n^* = a_{1n}(x,y)\cdot\exp[i\Phi_{1n}(x,y)] \\
\Phi_{1n}(x,y) = (\Phi_1 - \Phi_2)//2\pi = \left[ 2\pi\ \frac{Z(x,y)}{\Lambda_{1n}} \right]\ //\ 2\pi \\
\Lambda_{1n} = -\frac{\lambda_1\lambda_n}{\lambda_1 - \lambda_n} \\
Z_{1n}(x,y) = \Lambda_{1n}\ \frac{\Phi_{1n}(x,y)}{2\pi} \\
\delta Z_{1n} = \sqrt{2}\ \Lambda_{1n}\ \epsilon 
	\end{gather*}
	\begin{gather*}
n \ge 3:\quad Z_{12\_n}(x,y) = \textrm{round}\left[\frac{Z_{12\_(n-1)}(x,y)}{\Lambda_{1n}} \right]\cdot\Lambda_{1n} + Z_{1n}(x,y) \\
\delta Z_{12\_n} = \Delta\cdot\Lambda_{1n} + \delta Z_{1n} \\
Z_{12\_n} - Z_{12\_(n-1)} = s\ \Delta Z \\
\bar{Z}_{12\_n} = Z_{12\_n} - s\ \Lambda_{1n}\cdot(\Delta Z \gg \delta Z_{1n}) \\
\delta\bar{Z}_{12\_n} = \delta Z_{1n} = \sqrt{2}\cdot\Lambda_{1n}\epsilon
	\end{gather*}
\end{frame}


\begin{frame}[allowframebreaks]{h-MWOPU process}
	\begin{figure}
		\includegraphics[height=50 mm]{DH_tutorial_md.assets/image-20220427215653500}
	\end{figure}
	\begin{figure}
		\includegraphics[height=50 mm]{DH_tutorial_md.assets/image-20220427215846786}
	\end{figure}
	\begin{figure}
		\includegraphics[height=50 mm]{DH_tutorial_md.assets/image-20220427215914117}
	\end{figure}
\end{frame}


\begin{frame}{sample python code for h-MWOPU}
aaa
\end{frame}


\begin{frame}{W Osten, et al.: Direct shape measurement by digital wavefront reconstruction and multi-wavelength contouring}
	\vspace{-3 mm}
	\small Opt Engg 39 79 (2000)
	\begin{columns}
		\begin{column}{.6\textwidth}
			\begin{figure}
				\includegraphics[height=55 mm]{DH_tutorial_md.assets/image-20220427150423735}
			\end{figure}
		\end{column}
		\begin{column}{.4\textwidth}
			\begin{itemize}
				\item aaa
			\end{itemize}
		\end{column}
	\end{columns}
\end{frame}


%\begin{frame}{h-MWOPU example}
%Wagner, Chr., Osten, W., \& Seebacher, S. (2000). Direct shape measurement by digital wavefront reconstruction and multi-wavelength contouring. *Optical Engineering*, *39*(1), 79–85
%\begin{figure}
%	\includegraphics[height=50 mm]{DH_tutorial_md.assets/image-20220427150423735}
%\end{figure}
%\end{frame}


%\begin{frame}{aaa}
%	\vspace{-3 mm}
%	\small aaa
%	\begin{columns}
%		\begin{column}{.6\textwidth}
%			\begin{figure}
%				\includegraphics[height=55 mm]{DH_tutorial_md.assets/image-20220426135144501}
%			\end{figure}
%		\end{column}
%		\begin{column}{.4\textwidth}
%			\begin{itemize}
%				\item aaa
%			\end{itemize}
%		\end{column}
%	\end{columns}
%\end{frame}


\begin{frame}[allowframebreaks]{h-MWOPU example}
Wada, A., Kato, M., \& Ishii, Y. (2008). Large step-height measurements using multiple-wavelength holographic interferometry with tunable laser diodes. *JOURNAL OF THE OPTICAL SOCIETY OF AMERICA A-OPTICS IMAGE SCIENCE AND VISION*, *25*(12), 3013–3020
	\begin{figure}
		\includegraphics[height=40 mm]{DH_tutorial_md.assets/image-20220427172539858}
	\end{figure}
	\begin{figure}
		\includegraphics[height=50 mm]{DH_tutorial_md.assets/image-20220427172633113}
	\end{figure}
Object height calculated from $\Delta\phi_6(\Lambda_6=2.5\ \textrm{mm})$. (a) The entire distribution. (b) Plot of the object heights along the black line in panel (a) as a function of lateral position
\end{frame}


\begin{frame}{G Nardin, et al.: Versatile spectral modulation of a broadband source for digital holographic microscopy}
	\vspace{-3 mm}
	\small Opt Expr 24, 27791(2016)
	\begin{columns}
		\begin{column}{.6\textwidth}
			\begin{figure}
				\includegraphics[height=55 mm]{DH_tutorial_md.assets/image-20220506171904318}
			\end{figure}
		\end{column}
		\begin{column}{.4\textwidth}
			\begin{itemize}
				\item aaa
			\end{itemize}
		\end{column}
	\end{columns}
\end{frame}


%\begin{frame}{h-MWOPU example}
%Nardin, G., Colomb, T., Emery, Y., \& Moser, C. (2016). Versatile spectral modulation of a broadband source for digital holographic microscopy. *Optics Express*, *24*(24), 27791
%\begin{figure}
%	\includegraphics[height=50 mm]{DH_tutorial_md.assets/image-20220506171904318}
%\end{figure}
%\end{frame}


%\begin{frame}{aaa}
%	\vspace{-3 mm}
%	\small aaa
%	\begin{columns}
%		\begin{column}{.6\textwidth}
%			\begin{figure}
%				\includegraphics[height=55 mm]{DH_tutorial_md.assets/image-20220426135144501}
%			\end{figure}
%		\end{column}
%		\begin{column}{.4\textwidth}
%			\begin{itemize}
%				\item aaa
%			\end{itemize}
%		\end{column}
%	\end{columns}
%\end{frame}


\begin{frame}[allowframebreaks]{h-MWOPU example}
Kim, M. K. (2022). Phase microscopy and surface profilometry by digital holography. *Light: Advanced Manufacturing*, *3*(1), 1.
	\begin{figure}
		\includegraphics[height=40 mm]{DH_tutorial_md.assets/image-20220427214520872}
	\end{figure}
	\begin{figure}
		\includegraphics[height=50 mm]{DH_tutorial_md.assets/image-20220427214628422}
	\end{figure}
	\begin{figure}
		\includegraphics[height=50 mm]{DH_tutorial_md.assets/image-20220427214715153}
	\end{figure}
	\begin{figure}
		\includegraphics[height=50 mm]{DH_tutorial_md.assets/image-20220427215024923}
	\end{figure}
	\begin{figure}
		\includegraphics[height=50 mm]{DH_tutorial_md.assets/image-20220427215138931}
	\end{figure}
\end{frame}


\section{c-MWOPU: cumulative multi-wavelength optical phase unwrapping}
\begin{frame}[c]
	\centering\LARGE\textbf{\secname}
\end{frame}


\begin{frame}{c-MWOPU theory}
aaa
\end{frame}


\begin{frame}{c-MWOPU process}
aaa
\end{frame}


\begin{frame}{c-MWOPU sample python code}
aaa
\end{frame}


\begin{frame}{P Psota, et al.: Surface topography measurement by frequency sweeping digital holography}
	\vspace{-3 mm}
	\small Appl Opt 56, 7800 (2017)
	\begin{columns}
		\begin{column}{.6\textwidth}
			\begin{figure}
				\includegraphics[height=55 mm]{DH_tutorial_md.assets/image-20220429165841610}
			\end{figure}
		\end{column}
		\begin{column}{.4\textwidth}
			\begin{itemize}
				\item aaa
			\end{itemize}
		\end{column}
	\end{columns}
\end{frame}


%\begin{frame}{c-MWOPU example}
%Lédl, V., Psota, P., Kaván, F., Matoušek, O., \& Mokrý, P. (2017). Surface topography measurement by frequency sweeping digital holography. *Applied Optics*, *56*(28), 7808
%\begin{figure}
%	\includegraphics[height=50 mm]{DH_tutorial_md.assets/image-20220429165841610}
%\end{figure}
%\end{frame}


\section{h-MAOPU: hierarchical multi-angle optical phase unwrapping}
\begin{frame}[c]
	\centering\LARGE\textbf{\secname}
\end{frame}


\begin{frame}{multi-wavelength generation}
	\begin{itemize}
		\item discrete lasers
		\item tunable lsers
		\item index tuning
		\item angle scanning
	\end{itemize}
\end{frame}


\begin{frame}{theory of h-MAOPU}
aaa
\end{frame}


\begin{frame}{h-MAOPU process}
aaa
\end{frame}


\begin{frame}{sample python code}
aaa
\end{frame}


\begin{frame}{MK Kim: Phase microscopy and surface profilometry by digital holography}
	\vspace{-3 mm}
	\small Light: Adv Manufacturing 3, 1 (2022)
	\begin{columns}
		\begin{column}{.6\textwidth}
			\begin{figure}
				\includegraphics[height=40 mm]{DH_tutorial_md.assets/image-20220427215312948}
			\end{figure}
		\end{column}
		\begin{column}{.4\textwidth}
			\begin{itemize}
				\item aaa
			\end{itemize}
		\end{column}
	\end{columns}
\end{frame}


\begin{frame}{MK Kim: Phase microscopy and surface profilometry by digital holography}
	\vspace{-3 mm}
	\small Light: Adv Manufacturing 3, 1 (2022)
	\begin{columns}
		\begin{column}{.6\textwidth}
			\begin{figure}
				\includegraphics[height=40 mm]{DH_tutorial_md.assets/image-20220427215358192}
			\end{figure}
		\end{column}
		\begin{column}{.4\textwidth}
			\begin{itemize}
				\item aaa
			\end{itemize}
		\end{column}
	\end{columns}
\end{frame}


\begin{frame}{MK Kim: Phase microscopy and surface profilometry by digital holography}
	\vspace{-3 mm}
	\small Light: Adv Manufacturing 3, 1 (2022)
	\begin{columns}
		\begin{column}{.6\textwidth}
			\begin{figure}
				\includegraphics[height=40 mm]{DH_tutorial_md.assets/image-20220427215431949}
			\end{figure}
		\end{column}
		\begin{column}{.4\textwidth}
			\begin{itemize}
				\item aaa
			\end{itemize}
		\end{column}
	\end{columns}
\end{frame}


\begin{frame}{MK Kim: Phase microscopy and surface profilometry by digital holography}
	\vspace{-3 mm}
	\small Light: Adv Manufacturing 3, 1 (2022)
	\begin{columns}
		\begin{column}{.6\textwidth}
			\begin{figure}
				\includegraphics[height=40 mm]{DH_tutorial_md.assets/image-20220427215457707}
			\end{figure}
		\end{column}
		\begin{column}{.4\textwidth}
			\begin{itemize}
				\item aaa
			\end{itemize}
		\end{column}
	\end{columns}
\end{frame}


\begin{frame}{MK Kim: Phase microscopy and surface profilometry by digital holography}
	\vspace{-3 mm}
	\small Light: Adv Manufacturing 3, 1 (2022)
	\begin{columns}
		\begin{column}{.6\textwidth}
			\begin{figure}
				\includegraphics[height=40 mm]{DH_tutorial_md.assets/image-20220427215537750}
			\end{figure}
		\end{column}
		\begin{column}{.4\textwidth}
			\begin{itemize}
				\item aaa
			\end{itemize}
		\end{column}
	\end{columns}
\end{frame}


%\begin{frame}[allowframebreaks]{h-MAOPU example}
%Kim, M. K. (2022). Phase microscopy and surface profilometry by digital holography. *Light: Advanced Manufacturing*, *3*(1), 1.
%\begin{figure}
%	\includegraphics[height=40 mm]{DH_tutorial_md.assets/image-20220427215312948}
%\end{figure}
%\begin{figure}
%	\includegraphics[height=50 mm]{DH_tutorial_md.assets/image-20220427215358192}
%\end{figure}
%\begin{figure}
%	\includegraphics[height=50 mm]{DH_tutorial_md.assets/image-20220427215431949}
%\end{figure}
%\begin{figure}
%	\includegraphics[height=50 mm]{DH_tutorial_md.assets/image-20220427215457707}
%\end{figure}
%\begin{figure}
%	\includegraphics[height=50 mm]{DH_tutorial_md.assets/image-20220427215537750}
%\end{figure}
%\end{frame}


\begin{frame}{h-MAOPU example}
aaa
\end{frame}


\section{c-MAOPU: cumulative multi-angle optical phase unwrapping}
\begin{frame}[c]
	\centering\LARGE\textbf{\secname}
\end{frame}


\begin{frame}{c-MAOPU theory}
aaa
\end{frame}


\begin{frame}{c-MAOPU process}
aaa
\end{frame}


\begin{frame}{c-MAOPU sample python code}
aaa
\end{frame}


\begin{frame}{P Psota, et al.: Multiple angle digital holography for the shape measurement of the unpainted tympanic membrane}
	\vspace{-3 mm}
	\small Opt Expr  28, 24624 (2020)
	\begin{columns}
		\begin{column}{.6\textwidth}
			\begin{figure}
				\includegraphics[height=55 mm]{DH_tutorial_md.assets/image-20220502140028547}
			\end{figure}
		\end{column}
		\begin{column}{.4\textwidth}
			\begin{itemize}
				\item aaa
			\end{itemize}
		\end{column}
	\end{columns}
\end{frame}


%\begin{frame}{c-MAOPU example}
%Psota, P., Tang, H., Pooladvand, K., Furlong, C., Rosowski, J. J., Cheng, J. T., \& Lédl, V. (2020). Multiple angle digital holography for the shape measurement of the unpainted tympanic membrane. *Optics Express*, *28*(17), 24614
%\begin{figure}
%	\includegraphics[height=30 mm]{DH_tutorial_md.assets/image-20220502140028547}
%\end{figure}
%\end{frame}


\section{WSDIH: wavelength-scanning synthesized low-coherence holography}
\begin{frame}[c]
	\centering\LARGE\textbf{\secname}
\end{frame}


\begin{frame}{WSDIH theory}
aaa
\end{frame}


\begin{frame}{WSDIH process}
aaa
\end{frame}


\begin{frame}{WSDIH sample python code}
aaa
\end{frame}


\begin{frame}{JC Marron \& KS Schroeder: Holographic laser-radar}
	\vspace{-3 mm}
	\small Opt Lett 18 385 (1993)
	\begin{columns}
		\begin{column}{.6\textwidth}
			\begin{figure}
				\includegraphics[height=55 mm]{DH_tutorial_md.assets/image-20220502221331902}
			\end{figure}
		\end{column}
		\begin{column}{.4\textwidth}
			\begin{itemize}
				\item aaa
			\end{itemize}
		\end{column}
	\end{columns}
\end{frame}


%\begin{frame}{WSDIH example}
%MARRON, J. C., \& SCHROEDER, K. S. (1993). HOLOGRAPHIC LASER-RADAR. *OPTICS LETTERS*, *18*(5), 385–387
%\begin{figure}
%	\includegraphics[height=50 mm]{DH_tutorial_md.assets/image-20220502221331902}
%\end{figure}
%\end{frame}


\begin{frame}{A Dakoff, et al.: Microscopic three-dimensional imaging by digital interference holography}
	\vspace{-3 mm}
	\small J Elec. Imaging 12 643 (2003)
	\begin{columns}
		\begin{column}{.6\textwidth}
			\begin{figure}
				\includegraphics[height=55 mm]{DH_tutorial_md.assets/image-20220502222329616}
			\end{figure}
		\end{column}
		\begin{column}{.4\textwidth}
			\begin{itemize}
				\item aaa
			\end{itemize}
		\end{column}
	\end{columns}
\end{frame}


%\begin{frame}{WSDIH example}
%Dakoff, A., Gass, J., \& Kim, M. K. (2003). Microscopic three-dimensional imaging by digital interference holography. *JOURNAL OF ELECTRONIC IMAGING*, *12*(4), 643–647
%\begin{figure}
%	\includegraphics[height=50 mm]{DH_tutorial_md.assets/image-20220502222329616}
%\end{figure}
%\end{frame}


\begin{frame}{MC Potcoava, et al.: In vitro imaging of ophthalmic tissue by digital interference holography}
	\vspace{-3 mm}
	\small J Mod Opt 57 115 (2010)
	\begin{columns}
		\begin{column}{.6\textwidth}
			\begin{figure}
				\includegraphics[height=55 mm]{DH_tutorial_md.assets/image-20220502223428429}
			\end{figure}
		\end{column}
		\begin{column}{.4\textwidth}
			\begin{itemize}
				\item aaa
			\end{itemize}
		\end{column}
	\end{columns}
\end{frame}


%\begin{frame}{WSDIH example}
%Potcoava, M. C., Kay, C. N., Kim, M. K., \& Richards, D. W. (2010). In vitro imaging of ophthalmic tissue by digital interference holography. *JOURNAL OF MODERN OPTICS*, *57*(2, SI), 115–123
%\begin{figure}
%	\includegraphics[height=50 mm]{DH_tutorial_md.assets/image-20220502223428429}
%\end{figure}
%\end{frame}


\begin{frame}{C Depeursinge, et al.: Submicrometer tomography of cells by multiple-wavelength digital holographic microscopy in reflection}
	\vspace{-3 mm}
	\small Opt Lett 34 653 (2009)
	\begin{columns}
		\begin{column}{.6\textwidth}
			\begin{figure}
				\includegraphics[height=55 mm]{DH_tutorial_md.assets/image-20220502223113698}
			\end{figure}
		\end{column}
		\begin{column}{.4\textwidth}
			\begin{itemize}
				\item aaa
			\end{itemize}
		\end{column}
	\end{columns}
\end{frame}


%\begin{frame}{WSDIH example}
%Kuehn, J., Montfort, F., Colomb, T., Rappaz, B., Moratal, C., Pavillon, N., Marquet, P., \& Depeursinge, C. (2009). Submicrometer tomography of cells by multiple-wavelength digital holographic microscopy in reflection. *OPTICS LETTERS*, *34*(5), 653–655
%\begin{figure}
%	\includegraphics[height=50 mm]{DH_tutorial_md.assets/image-20220502223113698}
%\end{figure}
%\end{frame}


\begin{frame}{L Xu, et al.: High-precision three-dimensional shape reconstruction via digital refocusing in multi-wavelength digital holography}
	\vspace{-3 mm}
	\small Appl Opt 51, 2958 (2012)
	\begin{columns}
		\begin{column}{.6\textwidth}
			\begin{figure}
				\includegraphics[height=55 mm]{DH_tutorial_md.assets/image-20220506161850832}
			\end{figure}
		\end{column}
		\begin{column}{.4\textwidth}
			\begin{itemize}
				\item aaa
			\end{itemize}
		\end{column}
	\end{columns}
\end{frame}


%\begin{frame}{WSDIH example}
%Xu, L., Aleksoff, C. C., \& Ni, J. (2012). High-precision three-dimensional shape reconstruction via digital refocusing in multi-wavelength digital holography. *APPLIED OPTICS*, *51*(15), 2958–2967
%\begin{figure}
%	\includegraphics[height=50 mm]{DH_tutorial_md.assets/image-20220506161850832}
%\end{figure}
%\end{frame}


\begin{frame}{S Chen, et al.: Swept source digital holographic phase microscopy}
	\vspace{-3 mm}
	\small Opt Lett 41 665 (2016)
	\begin{columns}
		\begin{column}{.6\textwidth}
			\begin{figure}
				\includegraphics[height=55 mm]{DH_tutorial_md.assets/image-20220506162020326}
			\end{figure}
		\end{column}
		\begin{column}{.4\textwidth}
			\begin{itemize}
				\item aaa
			\end{itemize}
		\end{column}
	\end{columns}
\end{frame}


%\begin{frame}{WSDIH example}
%Chen, S., Ryu, J., Lee, K., \& Zhu, Y. (2016). Swept source digital holographic phase microscopy. *OPTICS LETTERS*, *41*(4), 665–668
%\begin{figure}
%	\includegraphics[height=35 mm]{DH_tutorial_md.assets/image-20220506162020326}
%\end{figure}
%\end{frame}


\section{ASDIH: angle-scanning synthesized low-coherence holography}
\begin{frame}[c]
	\centering\LARGE\textbf{\secname}
\end{frame}



\begin{frame}{ASDIH theory}
aaa
\end{frame}


\begin{frame}{ASDIH process}
aaa
\end{frame}


\begin{frame}{ASDIH sample python code}
aaa
\end{frame}


\begin{frame}{CK Hong, et al.: Optical section imaging of the tilted planes by illumination-angle-scanning digital interference holography}
	\vspace{-3 mm}
	\small Appl Opt 49, 5110 (2010)
	\begin{columns}
		\begin{column}{.6\textwidth}
			\begin{figure}
				\includegraphics[height=55 mm]{DH_tutorial_md.assets/image-20220429162140423}
			\end{figure}
		\end{column}
		\begin{column}{.4\textwidth}
			\begin{itemize}
				\item aaa
			\end{itemize}
		\end{column}
	\end{columns}
\end{frame}


%\begin{frame}{ASDIH example}
%Jeon, Y., \& Hong, C. K. (2010). Optical section imaging of the tilted planes by illumination-angle-scanning digital interference holography. *APPLIED OPTICS*, *49*(27), 5110–5116
%\begin{figure}
%	\includegraphics[height=50 mm]{DH_tutorial_md.assets/image-20220429162140423}
%\end{figure}
%\end{frame}

 
\begin{frame}{J Dong, et al.: Surface shape measurement by multi-illumination lensless Fourier transform digital holographic interferometry}
	\vspace{-3 mm}
	\small Opt Comm 402, 91 (2017)
	\begin{columns}
		\begin{column}{.6\textwidth}
			\begin{figure}
				\includegraphics[height=55 mm]{DH_tutorial_md.assets/image-20220429161726446}
			\end{figure}
		\end{column}
		\begin{column}{.4\textwidth}
			\begin{itemize}
				\item aaa
			\end{itemize}
		\end{column}
	\end{columns}
\end{frame}


%\begin{frame}{ASDIH example}
%Dong, J., Jia, S., \& Jiang, C. (2017). Surface shape measurement by multi-illumination lensless Fourier transform digital holographic interferometry. *Optics Communications*, *402*(May), 91–96
%\begin{figure}
%	\includegraphics[height=50 mm]{DH_tutorial_md.assets/image-20220429161726446}
%\end{figure}
%\end{frame}


\begin{frame}{T Kozacki, et al.: Multi-incidence digital holographic profilometry with high axial resolution and enlarged measurement range}
	\vspace{-3 mm}
	\small Opt Expr 28, 8185 (2020)
	\begin{columns}
		\begin{column}{.6\textwidth}
			\begin{figure}
				\includegraphics[height=55 mm]{DH_tutorial_md.assets/image-20220429160809187}
			\end{figure}
		\end{column}
		\begin{column}{.4\textwidth}
			\begin{itemize}
				\item aaa
			\end{itemize}
		\end{column}
	\end{columns}
\end{frame}


%\begin{frame}{ASDIH example}
%Martinez-Carranza, J., Mikuła-Zdańkowska, M., Ziemczonok, M., \& Kozacki, T. (2020). Multi-incidence digital holographic profilometry with high axial resolution and enlarged measurement range. *Optics Express*, *28*(6), 8185
%\begin{figure}
%	\includegraphics[height=50 mm]{DH_tutorial_md.assets/image-20220429160809187}
%\end{figure}
%\end{frame}


\section{further development}
\begin{frame}[c]
	\centering\LARGE\textbf{\secname}
\end{frame}


\begin{frame}{\secname}
	\begin{itemize}
		\item current state of the art
		\item current issues
		\item competing trending technologies
		\item potential solutions and advantages
	\end{itemize}
	\vspace{1 cm}
\end{frame}


\section{references}
\begin{frame}[c]
	\centering\LARGE\textbf{\secname}
\end{frame}


\begin{frame}{bibliography}
this is 2013 \cite{Kim2013} \\
this is 2020 ... \bibentry{Kim2020}
%this is 2020 ... \fullcite{Yu2014}
\bibliographystyle{bibstyles/osajnl}
%\bibliographystyle{plain}
%\bibliography{biblio.bib}
\end{frame}


\frame{}
\bibliography{biblio.bib}



\end{document}





