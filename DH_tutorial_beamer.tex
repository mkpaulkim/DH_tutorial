\documentclass[t, aspectratio=169]{beamer}
\usepackage[utf8]{inputenc}
\usepackage{amsmath}

\usetheme[progressbar=frametitle]{Madrid}
\usecolortheme{crane}
\setbeamertemplate{frame numbering}[fraction]

\usefonttheme[onlymath]{serif}
\usepackage{multicol}
\usepackage[absolute,overlay]{textpos}

\usepackage{bibentry}

%\title{Multi-Wavelength Techniques in Digital Holography: Tutorial}
%\author{Myung K Kim}
%\institute{Dept of Physics, University of South Florida, Tampa, FL USA 33620}
%\date{August 2022}

\title{Multi-Wavelength Techniques in Digital Holography}
\author{Myung K Kim}
\institute{University of South Florida}
\date{August 2022}


\begin{document}


\begin{frame}
	Optica: Digital Holography and Three-Dimensional Imaging\\
	Cambridge University, Cambridge, UK
	\titlepage
\end{frame}


\begin{frame}{abstract}
	\vspace{10 mm}
	\centering
Basic principles and techniques will be described for multi-wavelength or multi-illumination-angle mehtods for extending capabilities of digital holography for surface profilometry by optical phase unwrapping and topography/tomography by synthesized optical coherence
\end{frame}


\begin{frame}{acknowledgement}
	\begin{multicols}{2}
		\begin{itemize}
			\item{James Gass}
			\item{Aaron Dakof}
			\item{Dan Parshall}
			\item{Lingfeng Yu}
			\item{Chris Mann}
			\item{Alejandro Rostrepo}
			\item{Nilanthi Warnasooriya}
			\item{Alex Khmaladze}
			\item{Mariana Potcoava}
			\item{Leo Krzewina}
			\item{Bill Ash}
			\item{David Clark}
			\item{Xiao Yu}
			\item{Changgeng Liu}
			\item{Mauricio Flores}
			\item{Jisoo Hong}
			\item{Jiawen Weng}
			\item{Changwon Jang}
			\vspace{5 mm}
			\item{National Science Foundation}
			\item{National Eye Institute}
			\item{HoloVision}
			\item{HCube Optics}
		\end{itemize}
	\end{multicols}
\end{frame}


\begin{frame}[allowframebreaks]{Table of Contents}
	\tableofcontents[hideallsubsections]
\end{frame}


\section{digital holography}
\begin{frame}[c]
	\centering\LARGE\textbf{\secname}
\end{frame}


\begin{frame}{Denis Gabor: Microscopy by reconstructed wave-fronts}
	\vspace{-3 mm}
	\small Proc. Royal Soc London 197, 454 (1949)
	\begin{columns}
	    \begin{column}{.6\textwidth}
			\begin{figure}
				\includegraphics[height=55 mm]{DH_tutorial_md.assets/image-20220426134139308}
			\end{figure}
		\end{column}
		\begin{column}{.4\textwidth}
			\begin{itemize}
				\item aberrations in electron microscope
				\item optical demonstration
			\end{itemize}
		\end{column}
	\end{columns}
\end{frame}


\begin{frame}{EN Leigth \& J Upatnieks: Wavefront Reconstruction with Diffused Illumination and Three-Dimensional Objects}
	\vspace{-3 mm}
	\small J Opt Soc Am, 54, 1295(1964)
	\begin{columns}
		\begin{column}{.6\textwidth}
			\begin{figure}
				\includegraphics[height=55 mm]{DH_tutorial_md.assets/image-20220426135144501}
			\end{figure}
		\end{column}
		\begin{column}{.4\textwidth}
			\begin{itemize}
				\item laser: coherent illumination
				\item full 3D recording and reconstruction
			\end{itemize}
		\end{column}
	\end{columns}
\end{frame}


\begin{frame}{JW Goodman: Digital image formation from electronically detected holograms}
	\vspace{-3 mm}
	\small Appl Phys Lett 11, 77 (1967)
	\begin{columns}
		\begin{column}{.6\textwidth}
			\begin{figure}
				\includegraphics[height=55 mm]{DH_tutorial_md.assets/image-20220426135625638}
			\end{figure}
		\end{column}
		\begin{column}{.4\textwidth}
			\begin{itemize}
				\item aaa
			\end{itemize}
		\end{column}
	\end{columns}
\end{frame}


\begin{frame}{U Schnars \& W Jueptner: Direct recording of holograms by a CCD target and numerical reconstruction}
	\vspace{-3 mm}
	\small Opt Lett 24, 291 (1999)
	\begin{columns}
		\begin{column}{.6\textwidth}
			\begin{figure}
				\includegraphics[height=55 mm]{DH_tutorial_md.assets/image-20220426140019172}
			\end{figure}
		\end{column}
		\begin{column}{.4\textwidth}
			\begin{itemize}
				\item aaa
			\end{itemize}
		\end{column}
	\end{columns}
\end{frame}


\begin{frame}{C Depeursinge: Digital holography for quantitative phase-contrast imaging}
	\vspace{-3 mm}
	\small Opt Lett 24, 291 (1999)
	\begin{columns}
		\begin{column}{.6\textwidth}
			\begin{figure}
				\includegraphics[height=55 mm]{DH_tutorial_md.assets/image-20220426140050211}
			\end{figure}
		\end{column}
		\begin{column}{.4\textwidth}
			\begin{itemize}
				\item aaa
			\end{itemize}
		\end{column}
	\end{columns}
\end{frame}


\begin{frame}{holographic terms}
	\begin{figure}
		\includegraphics[width=4.0 in]{DH_tutorial_md.assets/image-20220415152931893}
	\end{figure}	
hologram recording \\
	\begin{center}
		$I = |E_R + E_O|^2 = |E_R|^2 + |E_O|^2 +E_R^*E_O + E_RE_O^*$
	\end{center}
	\pause
hologram reconstruction \\
	\begin{center}
		$E'_RI = E'_R|E_R+E_O|^2 
		=E'_R|E_R|^2 + E'_R|E_O|^2 + E'_RE_R^*E_O + E'_RE_RE_O^*$
	\end{center}
\end{frame}


\begin{frame}{holography of plane waves}
	\begin{figure}
		\includegraphics[width=4.0 in]{DH_tutorial_md.assets/diffract_planewave}
	\end{figure}
	\begin{itemize}
		\item aaa
	\end{itemize}
\end{frame}


\begin{frame}{holography of spherical waves}
	\begin{figure}
		\includegraphics[width=4.0 in]{DH_tutorial_md.assets/diffract_sphericalwave}
	\end{figure}
	\begin{itemize}
		\item aaa
	\end{itemize}
\end{frame}


\begin{frame}{diffraction from a 2D aperture}
	\begin{figure}
		\includegraphics[height=1.5 in]{DH_tutorial_md.assets/image-20220415162604622}
	\end{figure}
	\begin{gather*}
E(x,y;z) = -\frac{ik}{2\pi}\iint_{\Sigma_0} dx_0dy_0\,E_0(x_0,y_0)\frac{\exp(ikr)}{r}
	\end{gather*}
\end{frame}


\begin{frame}{Huygens convoluton method: HCM}	
	\begin{gather*}	
E(x,y;z) \approx -\frac{ik}{2\pi z} \iint_{\Sigma_0} dx_0dy_0\,E_0(x_0,y_0)\exp\left[ik\sqrt{(x-x_0)^2 + (y-y_0)^2 + z^2 }\right] \\
= E_0 \odot S_H 
	\end{gather*}
	\pause
	\begin{gather*}
\textbf{spherical wavefront} \qquad S_H(x,y;z) = - \frac{ik}{2\pi z} \exp\left[ik\sqrt{x^2+y^2+z^2}\right]
	\end{gather*}
\end{frame}


\begin{frame}{Fresnel transform method: FTM}	
	\vspace{-5 mm}
	\begin{gather*}	
r \approx z + \frac{x^2+y^2}{2z} \\
	\end{gather*}
	\pause
	\vspace{-8 mm}
	\begin{gather*}
E(x,y) = - \frac{ik}{2\pi z} \exp\left[ikz + \frac{ik}{2z} (x^2+y^2)\right] \iint dx_0\,dy_0E_0(x_0,y_0)\exp\left[\frac{ik}{2z}(x_0^2+y_0^2)\right] \\
\hspace{50mm}\times\exp\left[-\frac{ik}{z}(xx_0+yy_0)\right] \\ 
=\exp\left[\frac{ik}{2z}(x^2+y^2)\right]\mathfrak F\{E_0S_F\} \\
	\end{gather*}
	\pause
	\vspace{-8 mm}
	\begin{gather*}
\textbf{parabolic wave front} \qquad S_F(x,y;z) = -\frac{ik}{2\pi z}\exp\left[ikz + \frac{ik}{2z}(x^2+y^2)\right]
	\end{gather*}
\end{frame}


\begin{frame}{angular spectrum method: ASM}
	\begin{figure}
		\includegraphics[height=1.25 in]{DH_tutorial_md.assets/image-20220415153149971}
	\end{figure}
	\textbf{angular spectrum of input wave}: plane wave components
	\begin{gather*}
A_0(k_x,k_y) = \mathfrak F\left\{E_0(x_0,y_0)\right\}\left[k_x,k_y\right] \\
=\frac{1}{2\pi}\iint_{\Sigma_0} dx_0 dy_0 E_0(x_0,y_0) \exp\left[-i(k_x x_0 + k_y y_0)\right]
	\end{gather*}
\end{frame}


\begin{frame}{propagation of angular spectrum}
	\begin{figure}
		\includegraphics[height=1.25 in]{DH_tutorial_md.assets/image-20220415153149971}
	\end{figure}
\textbf{propagation of angular spectrum:} output wave is the sum of propagated plane wave components
	\begin{gather*}
E(x,y,z) = \mathfrak F^{-1}\left\{A_0(k_x,k_y) \exp[ik_z z]\right\}[x,y] \\
= \mathfrak F^{-1}\left\{\mathfrak F\{E_0\}\exp\left[i\sqrt{k^2 - k_x^2 - k_y^2}\ z\right]\right\} \\
k_z = \sqrt{k^2 - k_x^2 - k_y^2} \qquad \left[ k_x^2 + k_y^2 \le k^2 \right]	
	\end{gather*}
\end{frame}


\begin{frame}{FFT: fast Fourier transform}
	\begin{figure}
		\includegraphics[height=1.15 in]{DH_tutorial_md.assets/image-20220416111943805}
	\end{figure}
	\begin{gather*}
\mathfrak F\left\{f(x,y)\right\}[k_x,k_y] = \frac{1}{2\pi}\iint dx\,dy\,f(x,y)\exp\left[-i(k_x x + k_yy)\right] = F(k_x,k_y) \\
\mathfrak F^{-1}\left\{F(k_x,k_y)\right\}[x,y] = \frac{1}{2\pi}\iint dk_x\,dk_y\,F(k_x,k_y)\exp\left[+i(k_x x + k_yy)\right] = f(x,y) \\
K_x = 2\pi/dx \qquad dk_x = 2\pi/A_x = K_x/N_x \\
K_y = 2\pi/dy \qquad dk_y = 2\pi/A_y = K_y/N_y
	\end{gather*}
\end{frame}


\begin{frame}{range of recordable wavefront slope}
	\begin{itemize}
		\item $K_x/k=\lambda/dx$: maximum slope of wavefront before violating Nyquist
		\item $dk/k = \lambda/A_x$: minimum slope of wavefront to record one cycle over the frame size $A_x$ 
	\end{itemize}
\end{frame}


\begin{frame}{HCM vs ASM: propagation of spherical \& plane waves}
	\begin{columns}
		\begin{column}{.1\textwidth}
\textbf{ }\\
\vspace{10 mm}
\textbf{HCM}\\
\vspace{20 mm}
\textbf{ASM}
		\end{column}
		\begin{column}{.9\textwidth}
			\begin{figure}
				\includegraphics[height=50 mm]{DH_tutorial_md.assets/image-20220415160427601}
			\end{figure}
		\end{column}
	\end{columns}
\end{frame}


\begin{frame}{FTM vs HCM vs ASM}
\hspace{30 mm}\textbf{FTM}\hspace{30 mm}\textbf{HCM}\hspace{30 mm}\textbf{ASM}
	\begin{figure}
		\includegraphics[height=2.0 in]{DH_tutorial_md.assets/image-20220415160459990}
	\end{figure}
	\textbf{FTM}: 
	\[ Z_{\min} = \frac{X_0^2}{N\lambda}; \qquad\qquad Z_{\max} = \frac{X_0^2}{2\lambda} \]
\end{frame}


\begin{frame}[fragile]{sample Python code for AS diffraction}
	\begin{semiverbatim}
def asdiffract(EE):    
    xx, yy = meshgrid(-ax/2:dx:ax/2-dx, -ay/2:dy:ay/2-dy)
    k = 2*pi/wlen
    dkx, dky = 2*pi/ax, 2*pi/ay
    akx, aky = 2*pi/dx, 2*pi/dy
    kxx, kyy = meshgrid(-akx/2:dkx:akx/2-dkx, -aky/2:dky:aky/2-dky)
    
    FF = fftshift(fft(EE))    
    GG = exp(1j*z*sqrt(k**2 - kxx**2 - kyy**2))    
    HH = ifft(ifftshift(FF * GG))
    
    return HH    
	\end{semiverbatim}
\end{frame}


\section{digital holography system}
\begin{frame}[c]
	\centering\LARGE\textbf{\secname}
\end{frame}

\begin{frame}{DHM apparatus - reflection}
	\textbf{DHM: digital holographic microscopy}
	\begin{columns}
		\begin{column}{0.65\textwidth}
			\vspace{-5 mm}
			\begin{figure}
				\includegraphics[width=100 mm]{DH_tutorial_md.assets/DHM_optics_reflection}
			\end{figure}
		\end{column}
		\begin{column}{0.35\textwidth}
			\vspace{-5 mm}
			\begin{itemize}
				\begin{scriptsize}
					\item FC: fiber coupler
					\item MO: microscope objective
					\item L3: collimating lens
					\item MI: illumination mirror
					\item L1 \& L2: confocal lens pair
					\item PBS: polarizing beam splitter
					\item QO \& QR: quarter wave plates
					\item MR: piezo-mounted reference mirror
					\item L4 \& LC: camera imaging lens
					\item P: linear polarizer
					\item cam: camera
				\end{scriptsize}
			\end{itemize}
		\end{column}
	\end{columns}
\end{frame}


\begin{frame}{DHM apparatus - transmission}
	\begin{columns}
		\begin{column}{0.65\textwidth}
			\vspace{-5 mm}
			\begin{figure}
				\includegraphics[width=100 mm]{DH_tutorial_md.assets/DHM_optics_transmission}
			\end{figure}
		\end{column}
		\begin{column}{0.35\textwidth}
			\vspace{-5 mm}
			\begin{itemize}
				\begin{scriptsize}
					\item R: reticle attached to LED
					\item P1 \& P2: polarizers
					\item L1: collimating lens
					\item SO \& SR: object slide and reference compensating slide
					\item L2O \& L2R: illumination lens
					\item LO \& LR: microscope objectives
					\item BS1, BS2, BSO, \& BSR: polarizing beam splitter cubes
					\item QWO \& QWR: quarter wave plates
					\item L3: lens to adjust FOV
					\item MR: mirror mounted with Z-translation
					\item MO: mirror mounted with XY-tilt
				\end{scriptsize}
			\end{itemize}
		\end{column}
	\end{columns}
\end{frame}


\begin{frame}{Gabor holography}
	\hspace{30 mm} recording \hspace{50 mm} reconstruction
	\begin{figure}
		\includegraphics[width=120mm]{DH_tutorial_md.assets/image-20220426154854457}
	\end{figure}
	\begin{itemize}
		\item aaa
	\end{itemize}
\end{frame}


\begin{frame}{in-line holography}
	\hspace{30 mm} recording \hspace{50 mm} reconstruction
	\begin{figure}
		\includegraphics[width=120mm]{DH_tutorial_md.assets/image-20220426154916787}
	\end{figure}
	\begin{itemize}
		\item aaa
	\end{itemize}
\end{frame}


\begin{frame}{phase-shifting digital holography}
	\begin{columns}
		\begin{column}{0.6\textwidth}
			\begin{figure}
				\includegraphics[width=75 mm]{DH_tutorial_md.assets/image-20220426155721421}
			\end{figure}
		\end{column}
		\begin{column}{0.4\textwidth}
			\begin{itemize}
				\item aaa
			\end{itemize}
		\end{column}
	\end{columns}
\end{frame}


\begin{frame}{phase-shifting digital holography}
\[ \textbf{object:   } E_O(x,y) = \mathcal{E}_O(x,y)\exp[i\Phi(x,y)] \]
\[ \textbf{reference:   } E_n = \mathcal{E}_R \exp(i2\pi n/N) \]
	\pause
\[ I_n = |E_R + E_O|^2 = \mathcal{E}_R^2 + \mathcal{E}_O^2 + \mathcal{E}_R\mathcal{E}_O e^{i(2\pi n/N-\Phi)} + \mathcal{E}_R\mathcal{E}_O e^{i(-2\pi n/N+\Phi)} \]
	\pause
\[ S=\frac{1}{N}\sum_{n=0}^{N-1} I_n e^{i2\pi n/N} = \frac{1}{N}\left\{\left[\mathcal{E}_R^2 + \mathcal{E}_O^2 \right]\sum_n e^{i2\pi n/N} + \mathcal{E}_R\mathcal{E}_O e^{-i\Phi}\sum_n e^{i4\pi n/N} + \mathcal{E}_R\mathcal{E}_O e^{i\Phi}N \right\} \]
\[ = \mathcal{E}_R\mathcal{E}_O(x,y)\exp[i\Phi(x,y)] \]
	\pause
\[ E_O(x,y) = \frac{1}{N\mathcal E_R}\sum_{n=0}^{N-1} I_n \exp(i2\pi n/N) \]
\end{frame}


\begin{frame}{phase-shifting digital holography}
	\begin{columns}
		\begin{column}{0.5\textwidth}
			\vspace{-5 mm}
\[ \textrm{For  } N=4:\hspace{10 mm} \Phi(x,y) = \tan\frac{I_0 - I_2}{I_1 - I_3} \]	
			\begin{figure}
				\includegraphics[width=75 mm]{DH_tutorial_md.assets/image-20220426155750402}
			\end{figure}
		\end{column}
		\begin{column}{0.5\textwidth}		
		\begin{itemize}
			\item[a)] $ I_0 $
			\item[b)] $ I_1 $
			\item[c)] $ I_2 $
			\item[d)] $ I_3 $
			\item[e)] image reconstructed from $ I_O $
			\item[f)] $ \sqrt{E_O^2(x,y)}
			 $
			\item[g)] $ \Phi(x,y) $
			\item[h)] image reconstructed from $ \sqrt{E_O^2}\exp{i\Phi(x,y)} $
		\end{itemize}	
		\end{column}
	\end{columns}
\end{frame}


\begin{frame}[fragile]{sample Python code for PSDH}
	\begin{semiverbatim}
def psdh(HHH):
	ny, nx, nph = shape(HHH)
	EE = zeros(ny,nx)
	for n in range(nph):
		EE += HH[:,:,n] * exp(1j*n*2*pi/nph)
    EE = EE/nph
    return EE
	\end{semiverbatim}
\end{frame}


\begin{frame}{off-axis holography}
	\hspace{30 mm} recording \hspace{50 mm} reconstruction
	\begin{figure}
		\includegraphics[width=120mm]{DH_tutorial_md.assets/image-20220426154947369}
	\end{figure}
	\begin{itemize}
		\item aaa
	\end{itemize}
\end{frame}


\begin{frame}{spatial frequency bandwidth in off-axis holography}
	\begin{figure}
		\includegraphics[width=40 mm]{DH_tutorial_md.assets/angular_bandwidth}
	\end{figure}
	\begin{gather*}
\textbf{max object bandwidth:     } B = \frac{2}{2+3\sqrt{2}}K = \beta K \qquad \beta=0.32 \\
\textbf{optimum off-axis:   } K_0 = \frac{3}{2\sqrt{2}}B = \frac{3}{2\sqrt{2}+6} K = \gamma K \qquad \gamma=0.34 \qquad \sqrt{2}\,\gamma = 0.48 
	\end{gather*}
\end{frame}


\begin{frame}[fragile]{sample OXDH.py code}
	\begin{semiverbatim}
def oxdh(HH, qq):
    xx, yy = meshgrid(-ax/2:dx:ax/2-dx, -ay/2:dy:ay/2-dy)
    k = 2*pi/wlen
    dkx, dky = 2*pi/ax, 2*pi/ay
    akx, aky = 2*pi/dx, 2*pi/dy
    kxx, kyy = meshgrid(-akx/2:dkx:akx/2-dkx, -aky/2:dky:aky/2-dky)
    qx0, qy0, qx, qy = qq	% angular position & width
    kx0, ky0, kx, ky = k*sin(qx0), k*sin(qy0), k*sin(qx), k*sin(qy)   
    FF = fftshift(fft(HH))    
    BB = (((kxx-kx0)/kx)**2 + ((kyy-ky0)/ky)**2) < 1
    FF = FF * BB
    EE = ifft(ifftshift(FF)) 
    GG = exp(-1j*(kx0*xx + ky0*yy))
    EE = EE * GG    
    return EE
	\end{semiverbatim}
\end{frame}


\begin{frame}{CJ Mann, et al.: High-resolution quantitative phase-contrast microscopy by digital holography}
	\vspace{-3 mm}
	\small Opt Expr 13, 8693 (2005)
	\begin{columns}
		\begin{column}{.6\textwidth}
			\begin{figure}
				\includegraphics[height=55 mm]{DH_tutorial_md.assets/image-20220416215324683}
			\end{figure}
		\end{column}
		\begin{column}{.4\textwidth}
			\begin{itemize}
				\item aaa
			\end{itemize}
		\end{column}
	\end{columns}
\end{frame}


\begin{frame}{Depeursinge, et al.: Spatial analysis of erythrocyte membrane fluctuations by digital holographic microscopy}
	\vspace{-3 mm}
	\small Blood Cells, Molecules, and Diseases 42, 228–232 (2009)
	\begin{columns}
		\begin{column}{.6\textwidth}
			\begin{figure}
				\includegraphics[width=95 mm]{DH_tutorial_md.assets/image-20220427092446148}
			\end{figure}
		\end{column}
		\begin{column}{.4\textwidth}
			\begin{itemize}
				\item aaa
			\end{itemize}
		\end{column}
	\end{columns}
\end{frame}


\begin{frame}{Time-lapse of unstained, dividing and migrating cells}
	\vspace{-3 mm}
	\small wikipedia
	\begin{columns}
		\begin{column}{.6\textwidth}
			\begin{figure}
				\includegraphics[height=55 mm]{DH_tutorial_md.assets/DHM_cell_wiki.png}
			\end{figure}
		\end{column}
		\begin{column}{.4\textwidth}
			\begin{itemize}
				\item aaa
			\end{itemize}
		\end{column}
	\end{columns}
\end{frame}


\section{two-wavelength optical phase unwrapping}
\begin{frame}[c]
	\centering\LARGE\textbf{\secname}
\end{frame}


\begin{frame}{multi-wavelength DH: motivation}
multi-wavelength: motivation
	\begin{itemize}
		\item QPM with height/thickness range beyond wavelength
	\end{itemize}	
multi-wavelength: history
	\begin{itemize}
		\item HILDEBRAND, B. P., \& HAINES, K. A. (1967). MULTIPLE-WAVELENGTH AND MULTIPLE-SOURCE HOLOGRAPHY APPLIED TO CONTOUR GENERATION. *JOURNAL OF THE OPTICAL SOCIETY OF AMERICA*, *57*(2), 155+. https://doi.org/10.1364/JOSA.57.000155
		\item Cheng, Y.-Y., \& Wyant, J. C. (1984). Two-wavelength phase shifting interferometry. *Appl. Opt.*, *23*(24), 4539–4543. https://doi.org/10.1364/AO.23.004539
	\end{itemize}
wavelength-scanning: history
\end{frame}


\begin{frame}{QPM with single wavelength}
\[ E(x,y) = a(x,y)\cdot\exp[i\Phi(x,y)] \]
\pause
\[ \Phi(x,y) = \left[ 2\pi\ \frac{Z(x,y)}{\lambda} \right] \ //\ 2\pi \qquad \in [-\pi,\pi] \] \\
\[ a \ //\ b \quad \equiv \textrm{mod}\left(a+\frac{b}{2},\ b\right)-\frac{b}{2} \qquad \in \left[-\frac{b}{2}, \frac{b}{2}\right] \]
\pause
\[ Z_\lambda(x,y) = \lambda\ \frac{\Phi(x,y)}{2\pi} \qquad \in \left[-\frac{\lambda}{2}, \frac{\lambda}{2}\right] \]
\end{frame}


\begin{frame}{2WOPU}
\vspace{-5 mm}
\[ E_n(x,y) = a_n(x,y)\cdot\exp[i\Phi_n(x,y)] = a_n(x,y)\cdot\exp\left[ i2\pi\frac{Z(x,y)}{\lambda_n} \right] \]
\pause
\[ E_1\cdot E_2^* \equiv E_{12}(x,y) = a_{12}\cdot\exp[i\Phi_{12}(x,y)] \] \\
\[ \Phi_{12}(x,y) \equiv (\Phi_1 - \Phi_2) \ //\ 2\pi = \left[ 2\pi\ \frac{Z(x,y)}{\Lambda_{12}} \right]\ //\ 2\pi \qquad \in [-\pi,\pi] \]
\[ \Lambda_{12} = -\frac{\lambda_1 \lambda_2}{\lambda_1 - \lambda_2} \]
\pause
\[ Z_{12}(x,y) = \Lambda_{12}\cdot\frac{\Phi_{12}(x,y)}{2\pi} \qquad \in\left[-\frac{\Lambda_{12}}{2}, \frac{\Lambda_{12}}{2} \right] \]
\pause
\[ \frac{\Lambda}{\lambda} = \frac{\lambda}{\Delta\lambda} \]
\end{frame}


\begin{frame}{2WOPU process}
	\begin{columns}
		\begin{column}{0.6\textwidth}
			\vspace{-5 mm}
			\begin{figure}
				\includegraphics[height=70 mm]{DH_tutorial_md.assets/image-20220427215653500}
			\end{figure}
		\end{column}
		\begin{column}{0.4\textwidth}
			\begin{itemize}
				\item[a) ] $ Z(x) $
				\item[b) ] $ Z_1(x) $
				\item[c) ] $ Z_2(x) $
				\item[c) ] $ Z_{12}(x) $
			\end{itemize}
\vspace{5 mm}
$ Z_{\max}=15\ \mu m $
$ \lambda_1=0.635\ \mu m,\ \lambda_2 = 0.600\ \mu m $
$ \Lambda_{12} = 10.886\ \mu m $
$ \epsilon = 3\ \%,\ \delta Z_{12} = 0.462\ \mu m $
		\end{column}
	\end{columns}
\end{frame}


\begin{frame}[fragile]{sample 2WOPU.py code}
	\begin{semiverbatim}
aaa
	\end{semiverbatim}
\end{frame}


\begin{frame}{A Khmaladze, et al.: Phase imaging of cells by simultaneous dual-wavelength reflection digital holography}
	\vspace{-3 mm}
	\small Opt Expr 16, 10900 (2008)
	\begin{columns}
		\begin{column}{.6\textwidth}
			\begin{figure}
				\includegraphics[width= 90 mm]{DH_tutorial_md.assets/image-20220427141107699}
			\end{figure}
		\end{column}
		\begin{column}{.4\textwidth}
			\begin{itemize}
				\item aaa
			\end{itemize}
		\end{column}
	\end{columns}
\end{frame}


\begin{frame}{A Khmaladze, et al.: Phase imaging of cells by simultaneous dual-wavelength reflection digital holography}
	\vspace{-3 mm}
	\small Opt Expr 16, 10900 (2008)
	\begin{columns}
		\begin{column}{.6\textwidth}
			\begin{figure}
				\includegraphics[height=55 mm]{DH_tutorial_md.assets/image-20220427141311148}
			\end{figure}
		\end{column}
		\begin{column}{.4\textwidth}
			\begin{itemize}
				\item aaa
			\end{itemize}
		\end{column}
	\end{columns}
\end{frame}


\begin{frame}{MS Heimbeck, et al.: Terahertz digital holography using angular spectrum and dual wavelength reconstruction methods}
	\vspace{-3 mm}
	\small Opt Expr 19, 9192 (2011)
	\begin{columns}
		\begin{column}{.6\textwidth}
			\vspace{-5 mm}
			\begin{figure}
				\includegraphics[height=60 mm]{DH_tutorial_md.assets/image-20220427141355546}
			\end{figure}
		\end{column}
		\begin{column}{.4\textwidth}
			\begin{itemize}
				\item[a) ] object: plastic lens
				\item[b) ] hologram at 680 GHz
				\item[c) ] hologram at 725 GHz
				\item[d) \& e) ] amplitude reconstructions
				\item[f) \& g) ] pphase reconstructions
				\item[h) ] unwrapped reconstruction
				\item[i) ] cross-section through center of h) 
				\item[j) ] pseudo 3D rendering of h)
			\end{itemize}
		\end{column}
	\end{columns}
\end{frame}


\begin{frame}{MT Rinehart, et al.: Simultaneous two-wavelength transmission quantitative phase microscopy with a color camera}
	\vspace{-3 mm}
	\small Opt Lett 35, 2612 (2010)
	\begin{columns}
		\begin{column}{.6\textwidth}
			\begin{figure}
				\includegraphics[width=90 mm]{DH_tutorial_md.assets/image-20220427143118397}
			\end{figure}
		\end{column}
		\begin{column}{.4\textwidth}
			\begin{itemize}
				\item aaa
			\end{itemize}
		\end{column}
	\end{columns}
\end{frame}


\begin{frame}{MT Rinehart, et al.: Simultaneous two-wavelength transmission quantitative phase microscopy with a color camera}
	\vspace{-3 mm}
	\small Opt Lett 35, 2612 (2010)
	\begin{columns}
		\begin{column}{.6\textwidth}
			\vspace{-5 mm}
			\begin{figure}
				\includegraphics[width=70 mm]{DH_tutorial_md.assets/image-20220427143143483}
			\end{figure}
		\end{column}
		\begin{column}{.4\textwidth}
			\begin{itemize}
				\item[a) ] 532 nm OPD map after quality-map guided unwrapping
				\item[b) ] 532 nm OPD map after two-wavelength unwrapping
				\item[c) ] incorrect object height profile
				\item[d) ] object height profile from two-wavelength unwrapping
			\end{itemize}
		\end{column}
	\end{columns}
\end{frame}


\begin{frame}{Y Lee, et al.: Single-shot dual-wavelength phase unwrapping in parallel phase-shifting digital holography}
	\vspace{-3 mm}
	\small Opt Lett 39, 2374 (2014)
	\begin{columns}
		\begin{column}{.6\textwidth}
			\vspace{-5 mm}
			\begin{figure}
				\includegraphics[width=90 mm]{DH_tutorial_md.assets/image-20220427144722747}
			\end{figure}
		\end{column}
		\begin{column}{.4\textwidth}
			\begin{itemize}
				\item aaa
			\end{itemize}
		\end{column}
	\end{columns}
\end{frame}


\begin{frame}{Y Lee, et al.: Single-shot dual-wavelength phase unwrapping in parallel phase-shifting digital holography}
	\vspace{-3 mm}
	\small Opt Lett 39, 2374 (2014)
	\begin{columns}
		\begin{column}{.6\textwidth}
			\vspace{-5 mm}
			\begin{figure}
				\includegraphics[width=70 mm]{DH_tutorial_md.assets/image-20220427144827974}
			\end{figure}
		\end{column}
		\begin{column}{.4\textwidth}
			\begin{itemize}
				\item[a) ] parallel two-step phase-shifting interferometry
				\item[b) ] angular multi-plexing technique
			\end{itemize}
		\end{column}
	\end{columns}
\end{frame}


\section{h-MWOPU: hierarchical multi-wavelength optical phase unwrapping}
\begin{frame}[c]
	\centering\LARGE\textbf{\secname}
\end{frame}


\begin{frame}{noise in 2WOPU}
	\begin{gather*}
\delta\Phi_n = 2\pi\epsilon \\
\delta Z_n = \lambda_n\ \frac{\delta\Phi_n}{2\pi} = \lambda_n\epsilon \\
\delta\Phi_{12} = \sqrt{2}\ 2\pi\ \epsilon \\
\delta Z_{12} = \sqrt{2}\ \Lambda_{12}\ \epsilon \\
\frac{\delta Z_{12}}{\delta Z_n} = \frac{\sqrt{2}\ \Lambda_{12}}{\lambda_n}
	\end{gather*}
\end{frame}


\begin{frame}{3WOPU}
	\begin{columns}
		\begin{column}{.4\textwidth}
			\vspace{-5 mm}
			\begin{figure}
				\includegraphics[height=65 mm]{DH_tutorial_md.assets/sim_3wopu}
			\end{figure}
		\end{column}
		\begin{column}{.6\textwidth}
\vspace{-5 mm}
\begin{small}
\[ \lambda_1 = 0.635\ \mu m,\ \lambda_2=0.600\ \mu m,\ \lambda_3=0.530\ \mu m \]
\[ \Lambda_{12}=10.886\ \mu m,\ \Lambda_{13}=3.205\ \mu m \]
\end{small}
\pause
\[ Y_{12\_3} = \textrm{round}\left[\frac{Z_{12}}{\Lambda_{13}}\right]\Lambda_{13} \]
\[ Z'_{12\_3} = Y_{12} + Z_{13} \]
		\pause
\[ Z'_{12\_3} + \delta Z_{12\_3} = \textrm{round}\left[\frac{Z_{12} + \delta Z_{12}}{\Lambda_{13}} \right] \Lambda_{13} + (Z_{13} + \delta Z_{13}) \]
\[ \delta Z_{12\_3} = \Delta\cdot \Lambda_{13} + \delta Z_{13} = \Delta\cdot\Lambda_{13} + \sqrt{2}\ \Lambda_{13}\ \epsilon \] 
$\Delta\cdot\Lambda_{13}$: 		spikes of height $\Lambda_{13}$ scattered near $\textrm{round()}$ boundaries within $\delta Z_{12}$ 
		\end{column}
	\end{columns}
\end{frame}


\begin{frame}{noise reduction by 3WOPU}
	\begin{columns}
		\begin{column}{.4\textwidth}
			\vspace{-5 mm}
			\begin{figure}
				\includegraphics[height=65 mm]{DH_tutorial_md.assets/sim_3wopu}
			\end{figure}
		\end{column}
		\begin{column}{.6\textwidth}
\textbf{require: }
\[ \frac{\delta Z_{12}}{\Lambda_{13}} = \sqrt{2}\ \epsilon\ \frac{\Lambda_{12}}{\Lambda_{13}} \ll 1 \]
\[ \alpha = \frac{\Lambda_{13}}{\Lambda_{12}} \gg \sqrt{2}\ \epsilon \]
\pause
\textbf{despiking: }
\[ Z'_{12\_3} - Z_{12} = s\cdot\Delta Z \]
\[ Z_{12\_3} = Z'_{12\_3} - s\cdot\Lambda_{13}\cdot\left(\Delta Z \gg \delta Z_{13}\right)\  \]
\[ \delta Z_{12\_3} = \delta Z_{13} = \sqrt{2}\ \Lambda_{13}\ \epsilon \]
		\end{column}
	\end{columns}
\end{frame}


\begin{frame}{hierarchical MWOPU}
	\begin{columns}
		\begin{column}{0.4\textwidth}
			\vspace{-5 mm}
			\begin{figure}
				\includegraphics[height=65 mm]{DH_tutorial_md.assets/image-20220427215846786}
			\end{figure}
		\end{column}
		\begin{column}{0.6\textwidth}
\[ \{\lambda_n\} = \lambda_1,\ \lambda_2,\ \lambda_3,\ \cdots \]
\[ E_n(x,y) = a_n(x,y)\cdot\exp[i\Phi_n(x,y)] \]
\[ n \ge 2:\quad E_{1n}(x,y) = E_1\cdot E_n^* = a_{1n}(x,y)\cdot\exp[i\Phi_{1n}(x,y)] \]
\[ \Phi_{1n} = (\Phi_1 - \Phi_n)//2\pi = \left[ 2\pi\ \frac{Z}{\Lambda_{1n}} \right]\ //\ 2\pi \]
\[ \Lambda_{1n} = -\frac{\lambda_1\lambda_n}{\lambda_1 - \lambda_n} \]
\[ Z_{1n} = \Lambda_{1n}\ \frac{\Phi_{1n}}{2\pi} \]
\[ \delta Z_{1n} = \sqrt{2}\ \Lambda_{1n}\ \epsilon \]
		\end{column}
	\end{columns}
\end{frame}


\begin{frame}{hierarchical MWOPU}
	\begin{columns}
		\begin{column}{0.4\textwidth}
			\vspace{-5 mm}
			\begin{figure}
				\includegraphics[height=65 mm]{DH_tutorial_md.assets/image-20220427215846786}
			\end{figure}
		\end{column}
		\begin{column}{0.6\textwidth}
\[ n \ge 3:\quad Y_{12\_n}=\textrm{round}\left[\frac{Z_{12\_(n-1)}}{\Lambda_{1n}}\right]\Lambda_{1n} \]
\[ Z'_{12\_n} = Y_{12\_n} + Z_{1n} \]
\[ \delta Z_{12\_n} = \Delta\cdot\Lambda_{1n} + \delta Z_{1n} \]
\[ Z'_{12\_n} - Z_{12\_(n-1)} = s\cdot \Delta Z \]
\[ Z_{12\_n} = Z'_{12\_n} - s\cdot\Lambda_{1n}\cdot(\Delta Z \gg \delta Z_{1n}) \]
\[ \delta Z_{12\_n} = \delta Z_{1n} = \sqrt{2}\cdot\Lambda_{1n}\epsilon \]
\begin{small}
\hrulefill \\
$ \epsilon=3\ \%,\ \alpha=0.25 $ \\
$ \Lambda_{12}=2000\ \mu m,\ N=5,\ \Lambda_{15}=31.250\ \mu m $
\end{small}
		\end{column}
	\end{columns}
\end{frame}


\begin{frame}{h-MWOPU and stepping factor $\alpha$}
	\begin{columns}
		\begin{column}{0.7\textwidth}
			\vspace{-5 mm}
			\begin{figure}
				\includegraphics[height=60 mm]{DH_tutorial_md.assets/image-20220427215914117}
			\end{figure}
		\end{column}
		\begin{column}{0.3\textwidth}
\vspace{5 mm}
\textbf{large noise: } $\epsilon=7\ \% $ \\
$ $ \\
a) fast step: \\
$ \quad \alpha=0.25,\ N=5 $ \\ 
$ \quad\Lambda_{15}=31.25\ \mu m $ \\
$ $ \\
a) slow step: \\
$ \quad \alpha=0.50,\ N=8 $ \\ 
$ \quad\Lambda_{18}=31.25\ \mu m $ \\
		\end{column}
	\end{columns}
\end{frame}


\begin{frame}{sample python code for h-MWOPU}
aaa
\end{frame}


\begin{frame}{W Osten, et al.: Direct shape measurement by digital wavefront reconstruction and multi-wavelength contouring}
	\vspace{-3 mm}
	\small Opt Engg 39 79 (2000)
	\begin{columns}
		\begin{column}{.6\textwidth}
			\begin{figure}
				\includegraphics[width=70 mm]{DH_tutorial_md.assets/image-20220427150423735}
			\end{figure}
		\end{column}
		\begin{column}{.4\textwidth}
			\begin{itemize}
				\item aaa
			\end{itemize}
		\end{column}
	\end{columns}
\end{frame}


\begin{frame}{A Wada, et al.: Large step-height measurements using multiple-wavelength holographic interferometry with tunable laser diodes}
	\vspace{-3 mm}
	\small JOSA A 25, 3013 (2008)
	\begin{columns}
		\begin{column}{.45\textwidth}
			\begin{figure}
				\includegraphics[width= 70 mm]{DH_tutorial_md.assets/image-20220427172539858}
			\end{figure}
		\end{column}
		\begin{column}{.15\textwidth}
			\begin{figure}
				\includegraphics[width= 25 mm]{DH_tutorial_md.assets/image-20220427172633113}
			\end{figure}
		\end{column}
		\begin{column}{.4\textwidth}
			\begin{itemize}
				\item aaa
			\end{itemize}
		\end{column}
	\end{columns}
\end{frame}


\begin{frame}{G Nardin, et al.: Versatile spectral modulation of a broadband source for digital holographic microscopy}
	\vspace{-3 mm}
	\small Opt Expr 24, 27791(2016)
	\begin{columns}
		\begin{column}{.6\textwidth}
			\vspace{-5 mm}
			\begin{figure}
				\includegraphics[width=80 mm]{DH_tutorial_md.assets/2016_OX_24_27791}
			\end{figure}
		\end{column}
		\begin{column}{.4\textwidth}
			\begin{itemize}
				\item[a) ] $\lambda_1 = $ 650.3 nm, $ \lambda_2 = $ 559.7 nm, $ \lambda_3 = $ 575.3 nm, $ \Delta\lambda = $ 0.94 nm
				\item[b) ] $ \Phi_2 $
				\item[c) ] $ Z_1, Z_2, Z_3 $
			\end{itemize}
		\end{column}
	\end{columns}
\end{frame}


\begin{frame}{MK Kim: Phase microscopy and surface profilometry by digital holography}
	\vspace{-3 mm}
	\small Light: Adv Manufacturing, 3, 1 (2022)
	\begin{columns}
		\begin{column}{.47\textwidth}
			\begin{figure}
				\includegraphics[width=75 mm]{DH_tutorial_md.assets/image-20220427214520872}
			\end{figure}
		\end{column}
		\begin{column}{.28\textwidth}
			\begin{figure}
				\includegraphics[width=40 mm]{DH_tutorial_md.assets/image-20220427214628422}
			\end{figure}
		\end{column}
		\begin{column}{.25\textwidth}
			\begin{itemize}
				\item[a) ] $ \Phi_1(x,y) $
				\item[b) ] $ \Phi_2(x,y) $
				\item[c) ] $ Z_{12}(x,y) $
				\item[d) ] $ Z_{12\_3}(x,y) $
				\item[e) ] $ Z_{12\_4}(x,y) $
				\item[f) ] $ Z_{\textrm{proc}}(x,y) $
			\end{itemize}
		\end{column}
	\end{columns}
\end{frame}


\begin{frame}{MK Kim: Phase microscopy and surface profilometry by digital holography}
	\vspace{-3 mm}
	\small Light: Adv Manufacturing, 3, 1 (2022)
	\begin{columns}
		\begin{column}{.47\textwidth}
			\begin{figure}
				\includegraphics[width=75 mm]{DH_tutorial_md.assets/image-20220427214715153}
			\end{figure}
		\end{column}
		\begin{column}{.28\textwidth}
			\begin{figure}
				\includegraphics[width=40 mm]{DH_tutorial_md.assets/image-20220427215024923}
			\end{figure}
		\end{column}
		\begin{column}{.25\textwidth}
			\begin{itemize}
				\item[a) ] $ \Phi_1(x,y) $
				\item[b) ] $ \Phi_2(x,y) $
				\item[c) ] $ Z_{12}(x,y) $
				\item[d) ] $ Z_{12\_3}(x,y) $
				\item[e) ] $ Z_{12\_4}(x,y) $
				\item[f) ] $ Z_{\textrm{proc}}(x,y) $
			\end{itemize}
		\end{column}
	\end{columns}
\end{frame}


\begin{frame}{MK Kim: Phase microscopy and surface profilometry by digital holography}
	\vspace{-3 mm}
	\small Light: Adv Manufacturing, 3, 1 (2022)
	\begin{figure}
		\includegraphics[height=60 mm]{DH_tutorial_md.assets/image-20220427215138931_a}
	\end{figure}
\end{frame}


\section{c-MWOPU: cumulative multi-wavelength optical phase unwrapping}
\begin{frame}[c]
	\centering\LARGE\textbf{\secname}
\end{frame}


\begin{frame}{c-MWOPU theory}
%\[ \Phi_n = \left[ 2\pi\frac{Z}{\lambda_n} \right]//2\pi \]
%\[ \Phi_{12} = \Phi_1 - \Phi_2 = 2\pi Z\left[ \frac{1}{\lambda_1} - \frac{1}{\lambda_2} \right] = 2\pi\frac{Z}{\Lambda_{12}} \hspace{5 mm} [\Lambda_{12} > Z_{\max}] \]
%\[ Z_{12} = \Lambda_{12}\frac{\Phi_{12}}{2\pi} \]
Choose $\displaystyle\frac{1}{\lambda_n}$ at uniform intervals:
	\begin{small}
\[ \frac{1}{\lambda_1}-\frac{1}{\lambda_2} = \frac{1}{\lambda_2}-\frac{1}{\lambda_3} = \cdots \equiv \frac{1}{\Lambda} \]
\[ \Phi_{12} = \Phi_1-\Phi_2 = 2\pi Z\left[ \frac{1}{\lambda_1}-\frac{1}{\lambda_2} \right] = 2\pi\frac{Z}{\Lambda} \hspace{5 mm} [\Lambda_{12}>Z_{\max}] \]
\[ \Phi_{23} = \Phi_2-\Phi_3 = 2\pi Z\left[ \frac{1}{\lambda_2}-\frac{1}{\lambda_3} \right] = 2\pi\frac{Z}{\Lambda} \]
\[ \cdots \]
\[ \Phi_{n,n+1} = \Phi_n-\Phi_{n+1} = 2\pi Z\left[ \frac{1}{\lambda_n}-\frac{1}{\lambda_{n+1}} \right] = 2\pi\frac{Z}{\Lambda} \]
\[ \Psi_{1,n+1} = \Phi_{12}+\Phi_{23}+\cdots+\Phi_{n,n+1} = \Phi_1-\Phi_{n+1} = 2\pi Z\left[\frac{1}{\lambda_1}-\frac{1}{\lambda_{n+1}}\right] = 2\pi n\frac{Z}{\Lambda} \]
	\end{small}
%\[ Z_{1\_n} = \frac{\Lambda}{n}\frac{\Psi_{1,n+1}}{2\pi} \]
\end{frame}


\begin{frame}{c-MWOPU theory}
\[ \Psi_{1,n+1} = 2\pi n\frac{Z}{\Lambda} \]
\[ Z[\Psi_{1,n+1}] = \frac{\Lambda}{n}\frac{\Psi_{1,n+1}}{2\pi} \]
noise in $Z[\Psi_{1,n+1}]$
\[ \delta Z[\Psi_{1,n+1}] = \frac{\Lambda}{2\pi n}\sqrt{2}\ 2\pi\epsilon = \frac{\sqrt{2}\epsilon\Lambda}{n} = \frac{\delta Z_{12}}{n} \]
\end{frame}


\begin{frame}{c-MWOPU process}
aaa
\end{frame}


\begin{frame}{c-MWOPU sample python code}
aaa
\end{frame}


\begin{frame}{P Psota, et al.: Surface topography measurement by frequency sweeping digital holography}
	\vspace{-3 mm}
	\small Appl Opt 56, 7800 (2017)
	\begin{columns}
		\begin{column}{.6\textwidth}
			\begin{figure}
				\includegraphics[height=55 mm]{DH_tutorial_md.assets/image-20220429165841610}
			\end{figure}
		\end{column}
		\begin{column}{.4\textwidth}
			\begin{itemize}
				\item aaa
			\end{itemize}
		\end{column}
	\end{columns}
\end{frame}


\section{WSDIH: wavelength-scanning synthesized low-coherence holography}
\begin{frame}[c]
	\centering\LARGE\textbf{\secname}
\end{frame}


\begin{frame}{WSDIH theory}
object field at $\mathbf r_P$: $ A(\mathbf r_P) $ \\
Huygens wavelet arriving at $\mathbf r$: $ A(\mathbf r_P)\exp(ik|\mathbf r - \mathbf r_P|) $ \\
Field at $\mathbf r$ for extended object: 
\[ E_k(\mathbf r) \sim \int A(\mathbf r_P)\exp(ik|\mathbf r - \mathbf r_P|)\ d^3\mathbf r_P \]
Use a series of $k$'s and sum over all $E_k$'s: 
\[ E(\mathbf r) \sim \sum_k E_k(\mathbf r_P) \sim \sum_k \int A(\mathbf r_P)\exp(ik|\mathbf r - \mathbf r_P|)d^3\ \mathbf r_P \]
\[ \sim \int A(\mathbf r_P)\delta(\mathbf r - \mathbf r_P)\ d^3\mathbf r_P \]
\[ \sim A(\mathbf r) \]
\end{frame}


\begin{frame}{WSDIH process}
	\begin{figure}
		\includegraphics[height=65 mm]{DH_tutorial_md.assets/fig_12-28}
	\end{figure}
\end{frame}


\begin{frame}{WSDIH process}
	\begin{figure}
		\includegraphics[height=50 mm]{DH_tutorial_md.assets/fig_12-29}
	\end{figure}
	\begin{itemize}
 		\setlength{\itemindent}{15 mm}
		\item[a) ] build-up of axial resolution with n = 1, 2, 4, 8, 20
		\item[b) ] contour images of coin at 60 $\mu m$ axial intervals
	\end{itemize}
\end{frame}


%\begin{frame}{WSDIH theory}
%aaa
%\end{frame}


%\begin{frame}{WSDIH process}
%aaa
%\end{frame}


\begin{frame}{WSDIH sample python code}
aaa
\end{frame}


\begin{frame}{JC Marron \& KS Schroeder: Holographic laser-radar}
	\vspace{-3 mm}
	\small Opt Lett 18 385 (1993)
	\begin{columns}
		\begin{column}{.6\textwidth}
			\begin{figure}
				\includegraphics[width=50 mm]{DH_tutorial_md.assets/image-20220502221331902}
			\end{figure}
		\end{column}
		\begin{column}{.4\textwidth}
			\begin{itemize}
				\item aaa
			\end{itemize}
		\end{column}
	\end{columns}
\end{frame}


\begin{frame}{A Dakoff, et al.: Microscopic three-dimensional imaging by digital interference holography}
	\vspace{-3 mm}
	\small J Elec. Imaging 12 643 (2003)
	\begin{columns}
		\begin{column}{.6\textwidth}
			\begin{figure}
				\includegraphics[height=55 mm]{DH_tutorial_md.assets/image-20220502222329616}
			\end{figure}
		\end{column}
		\begin{column}{.4\textwidth}
			\begin{itemize}
				\item[a) ] numeral 2 in the 2000 mintage mark of a penny, 1 x 1 mm$^2$
				\item[b) ] fire ant's compound eye, 1 x 1 mm$^2$
			\end{itemize}
		\end{column}
	\end{columns}
\end{frame}


\begin{frame}{MC Potcoava, et al.: In vitro imaging of ophthalmic tissue by digital interference holography}
	\vspace{-3 mm}
	\small J Mod Opt 57 115 (2010)
	\begin{columns}
		\begin{column}{.6\textwidth}
			\begin{figure}
				\includegraphics[width= 90 mm]{DH_tutorial_md.assets/image-20220502223428429}
			\end{figure}
		\end{column}
		\begin{column}{.4\textwidth}
			Reconstructed volume of the human optic nerve sample
			\begin{itemize}
				\item[a) ] FOV = 1100 x 1100 $\mu m^2 $
				\item[b) ] y-z cross-sections, 1100 x 280 $\mu m^2 $
				\item[c) ] x-z cross-sections, 280 x 1100 $\mu m^2 $
			\end{itemize}
		\end{column}
	\end{columns}
\end{frame}


\begin{frame}{C Depeursinge, et al.: Submicrometer tomography of cells by multiple-wavelength digital holographic microscopy in reflection}
	\vspace{-3 mm}
	\small Opt Lett 34 653 (2009)
	\begin{columns}
		\begin{column}{.6\textwidth}
			\begin{figure}
				\includegraphics[height=55 mm]{DH_tutorial_md.assets/image-20220502223113698}
			\end{figure}
		\end{column}
		\begin{column}{.4\textwidth}
			\begin{itemize}
				\item aaa
			\end{itemize}
		\end{column}
	\end{columns}
\end{frame}


\begin{frame}{L Xu, et al.: High-precision three-dimensional shape reconstruction via digital refocusing in multi-wavelength digital holography}
	\vspace{-3 mm}
	\small Appl Opt 51, 2958 (2012)
	\begin{columns}
		\begin{column}{.6\textwidth}
			\begin{figure}
				\includegraphics[width= 90 mm]{DH_tutorial_md.assets/image-20220506161850832}
			\end{figure}
		\end{column}
		\begin{column}{.4\textwidth}
			\begin{itemize}
				\item aaa
			\end{itemize}
		\end{column}
	\end{columns}
\end{frame}


\begin{frame}{S Chen, et al.: Swept source digital holographic phase microscopy}
	\vspace{-3 mm}
	\small Opt Lett 41 665 (2016)
	\begin{columns}
		\begin{column}{.6\textwidth}
			\begin{figure}
				\includegraphics[width= 90 mm]{DH_tutorial_md.assets/image-20220506162020326}
			\end{figure}
		\end{column}
		\begin{column}{.4\textwidth}
			\begin{itemize}
				\item aaa
			\end{itemize}
		\end{column}
	\end{columns}
\end{frame}


\section{h-MAOPU: hierarchical multi-angle optical phase unwrapping}
\begin{frame}[c]
	\centering\LARGE\textbf{\secname}
\end{frame}


\begin{frame}{multi-wavelength generation}
	\begin{itemize}
		\item discrete lasers
		\item tunable lsers
		\item index tuning
		\item angle scanning
	\end{itemize}
\[ \lambda_\theta = \lambda\cdot\cos\theta \]
\end{frame}


\begin{frame}{theory of h-MAOPU}
aaa
\end{frame}


\begin{frame}{h-MAOPU process}
aaa
\end{frame}


\begin{frame}{sample python code}
aaa
\end{frame}


\begin{frame}{MK Kim: Phase microscopy and surface profilometry by digital holography}
	\vspace{-3 mm}
	\small Light: Adv Manufacturing, 3, 1 (2022)
	\begin{columns}
		\begin{column}{.47\textwidth}
			\begin{figure}
				\includegraphics[width=75 mm]{DH_tutorial_md.assets/image-20220427215312948}
			\end{figure}
		\end{column}
		\begin{column}{.28\textwidth}
			\begin{figure}
				\includegraphics[width=40 mm]{DH_tutorial_md.assets/image-20220427215358192}
			\end{figure}
		\end{column}
		\begin{column}{.25\textwidth}
			\begin{itemize}
				\item[a) ] $ \Phi_1(x,y) $
				\item[b) ] $ \Phi_2(x,y) $
				\item[c) ] $ Z_{12}(x,y) $
				\item[d) ] $ Z_{12\_3}(x,y) $
				\item[e) ] $ Z_{12\_8}(x,y) $
				\item[f) ] $ Z_{\textrm{proc}}(x,y) $
			\end{itemize}
		\end{column}
	\end{columns}
\end{frame}


\begin{frame}{MK Kim: Phase microscopy and surface profilometry by digital holography}
	\vspace{-3 mm}
	\small Light: Adv Manufacturing, 3, 1 (2022)
	\begin{columns}
		\begin{column}{.47\textwidth}
			\begin{figure}
				\includegraphics[width=75 mm]{DH_tutorial_md.assets/image-20220427215431949}
			\end{figure}
		\end{column}
		\begin{column}{.28\textwidth}
			\begin{figure}
				\includegraphics[width=40 mm]{DH_tutorial_md.assets/image-20220427215457707}
			\end{figure}
		\end{column}
		\begin{column}{.25\textwidth}
			\begin{itemize}
				\item[a) ] $ \Phi_1(x,y) $
				\item[b) ] $ \Phi_2(x,y) $
				\item[c) ] $ Z_{12}(x,y) $
				\item[d) ] $ Z_{12\_3}(x,y) $
				\item[e) ] $ Z_{12\_8}(x,y) $
				\item[f) ] $ Z_{\textrm{proc}}(x,y) $
			\end{itemize}
		\end{column}
	\end{columns}
\end{frame}


\begin{frame}{MK Kim: Phase microscopy and surface profilometry by digital holography}
	\vspace{-3 mm}
	\small Light: Adv Manufacturing, 3, 1 (2022)
	\begin{columns}
		\begin{column}{.47\textwidth}
%			\begin{figure}
%				\includegraphics[width=75 mm]{DH_tutorial_md.assets/image-20220427215431949}
%			\end{figure}
		\end{column}
		\begin{column}{.28\textwidth}
			$ \alpha=0.125 $ \\ instead of $ \alpha = 0.25 $
			\begin{figure}
				\includegraphics[width=40 mm]{DH_tutorial_md.assets/image-20220427215537750}
			\end{figure}
		\end{column}
		\begin{column}{.25\textwidth}
			\begin{itemize}
				\item[a) ] $ Z_{12}(x,y) $
				\item[b) ] $ Z_{12\_3}(x,y) $
				\item[c) ] $ Z_{12\_8}(x,y) $
				\item[d) ] $ Z_{\textrm{proc}}(x,y) $
			\end{itemize}
		\end{column}
	\end{columns}
\end{frame}


\section{c-MAOPU: cumulative multi-angle optical phase unwrapping}
\begin{frame}[c]
	\centering\LARGE\textbf{\secname}
\end{frame}


\begin{frame}{c-MAOPU theory}
aaa
\end{frame}


\begin{frame}{c-MAOPU process}
aaa
\end{frame}


\begin{frame}{c-MAOPU sample python code}
aaa
\end{frame}


\begin{frame}{P Psota, et al.: Multiple angle digital holography for the shape measurement of the unpainted tympanic membrane}
	\vspace{-3 mm}
	\small Opt Expr  28, 24624 (2020)
	\begin{columns}
		\begin{column}{.6\textwidth}
			\begin{figure}
%				\includegraphics[width=90 mm]{DH_tutorial_md.assets/image-20220502140028547}
				\includegraphics[width=70 mm]{DH_tutorial_md.assets/psota_tympanic}
			\end{figure}
		\end{column}
		\begin{column}{.4\textwidth}
			\begin{itemize}
				\item aaa
			\end{itemize}
		\end{column}
	\end{columns}
\end{frame}


\section{ASDIH: angle-scanning synthesized low-coherence holography}
\begin{frame}[c]
	\centering\LARGE\textbf{\secname}
\end{frame}



\begin{frame}{ASDIH theory}
aaa
\end{frame}


\begin{frame}{ASDIH process}
aaa
\end{frame}


\begin{frame}{ASDIH sample python code}
aaa
\end{frame}


\begin{frame}{CK Hong, et al.: Optical section imaging of the tilted planes by illumination-angle-scanning digital interference holography}
	\vspace{-3 mm}
	\small Appl Opt 49, 5110 (2010)
	\begin{columns}
		\begin{column}{.6\textwidth}
			\begin{figure}
				\includegraphics[height=55 mm]{DH_tutorial_md.assets/image-20220429162140423}
			\end{figure}
		\end{column}
		\begin{column}{.4\textwidth}
			\begin{itemize}
				\item aaa
			\end{itemize}
		\end{column}
	\end{columns}
\end{frame}

 
\begin{frame}{J Dong, et al.: Surface shape measurement by multi-illumination lensless Fourier transform digital holographic interferometry}
	\vspace{-3 mm}
	\small Opt Comm 402, 91 (2017)
	\begin{columns}
		\begin{column}{.6\textwidth}
			\begin{figure}
				\includegraphics[height=55 mm]{DH_tutorial_md.assets/image-20220429161726446}
			\end{figure}
		\end{column}
		\begin{column}{.4\textwidth}
			\begin{itemize}
				\item aaa
			\end{itemize}
		\end{column}
	\end{columns}
\end{frame}


\begin{frame}{T Kozacki, et al.: Multi-incidence digital holographic profilometry with high axial resolution and enlarged measurement range}
	\vspace{-3 mm}
	\small Opt Expr 28, 8185 (2020)
	\begin{columns}
		\begin{column}{.6\textwidth}
			\begin{figure}
				\includegraphics[width= 90 mm]{DH_tutorial_md.assets/image-20220429160809187}
			\end{figure}
		\end{column}
		\begin{column}{.4\textwidth}
			\begin{itemize}
				\begin{scriptsize}
					\item[a) ] tall object, 4 um per step
					\item[b) ] horizontal cross-sections A-A \& B-B of tall object
					\item[c) ] vertical cross-sections C-C, D-D, E-E, F-F of tall object
					\item[d) ] low object, 0.5 um per step
					\item[e) ] horizontal cross-sections A-A \& B-B of low object
					\item[f) ] vertical cross-sections C-C, D-D, E-E, F-F of low object
				\end{scriptsize}
			\end{itemize}
		\end{column}
	\end{columns}
\end{frame}


\section{further development}
\begin{frame}[c]
	\centering\LARGE\textbf{\secname}
\end{frame}


\begin{frame}{\secname}
	\begin{itemize}
		\item current state of the art
		\item current issues
		\item competing trending technologies
		\item potential solutions and advantages
	\end{itemize}
	\vspace{1 cm}
\end{frame}


\end{document}





